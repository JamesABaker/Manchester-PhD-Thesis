
\chapter{The ``negative-not-inside'' rule}
\section{Abstract}

\section{Introduction}
As the idea of positive residues inside the cytoplasm emerged during the late 1980s, so did the idea of negative residues working in concert with \gls{tmh} orientation. It was shown that removing a single lysine residue reversed the topology of a model Escherichia coli protein, whereas much higher numbers of negatively charged residues are needed to reverse topology \cite{Nilsson1990}. One would also expect to see a skew in negatively charged distribution if a cooperation between oppositely charged residues orientated a \gls{tmh}, however there is no conclusive evidence in the literature for an opposing negatively charged skew \cite{Granseth2005, Nilsson2005, Sharpe2010, Baeza-Delgado2013, Pogozheva2013}. However, in {\it E. coli} negative residues do experience electrical pulling forces when travelling through the SecYEG translocon indicating that negative charges are biologically relevant \cite{Ismail2015}.

\section{Methods}

\subsection{Normalisation}

$c_r=\frac{(a_{K,r}+a_{R,r})-(a_{D,r}+a_{E,r})}{N}$

$p_{i,r}=\frac{a_{i,r}}{\underset{r}{\max}{(a_r)}}$

$q_{i,r}=\frac{100·a_{i,r}}{a_i}$

\section{Biophysicochemical differences in multi-pass and single-pass helices}
