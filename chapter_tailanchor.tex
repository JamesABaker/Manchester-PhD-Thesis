\chapter{Tail-Anchored Protein Datasets}
\sloppy
% gls refers to a glossary term, cite refers to an entry from the separate bibtex folder. url is a lazy way of forcing monospaced text. textit is italics. sections and subsections are the headers.
\section{Abstract}

\section{Introduction}

~\gls{ta} proteins are a topologically distinct class of intracellular proteins defined by their single carboxy-terminal~\gls{tms} with a cytosolic facing amino-terminus.
~\gls{ta} proteins are involved in a range of key cellular functions including protein translocation and apoptosis.
Additionally, within the~\gls{ta} class of proteins are a set of vesicle fusion proteins called~\gls{snare} proteins.
There is biomedical interest in~\gls{snare} drug delivery mechanisms.
~\gls{snare}s can fuse liposomes containing various drug payloads into the membrane.

%Evidence for completely unassissted insertion Kemper et al., 2008; Meineke et al., 2008; Setoguchi et al., 2006
%Alternative mechanisms (Kutay et al., 1995; Nyathi et al., 2013; Chartron et al., 2012)
% a novel SRP- independent (SND) pathway in yeast (Aviram et al., 2016).

\gls{ta} proteins are integrated into the~\gls{er} membrane by different molecular machinery than the co-translational proteins.
These mechanisms include the Hsp70/Hsc40 chaperone system~\cite{Rabu2008}, the TRC40/Get3 ATP dependent chaperone system~\cite{Johnson2013, Chartron2012, Wang2014},


The pipeline generates a list of singlepass proteins with a transmembrane domain close to the C terminal, that are not splice isoforms.
A previous study by Kalbfleisch \textit{et al.} published in Traffic 2007 (8: 1687-1694) predicted 411 tail anchor proteins~\cite{Kalbfleisch2007}.
The tools developed herein are openly available for re-application to other datasets.
Notably, known~\gls{snare} transmembrane helices are highly hydrophobic even compared to other~\gls{ta} transmembrane helices.
We compare Kyte and Doolittle hydrophobicity profiles of our filtered human protein list against the profiles of previously known~\gls{snare} and~\gls{ta} proteins.
This provided a list of potential~\gls{snare} proteins in addition to potential spontaneously inserting~\gls{ta} proteins similar to cytochrome b5 which have the least hydrophobic transmembrane helices.

Tail-anchored proteins are a topologically distinct class of intracellular proteins defined by their single carboxy-terminal transmembrane domain with a cytosolic-facing amino-terminus.
%Need a figure here of a crystal structure TA.

Tail-anchored proteins are involved in a range of key cellular functions including protein translocation and apoptosis.
Additionally, within the tail-anchored class of proteins are a set of vesicle fusion proteins called \gls{snare} proteins.
There is biomedical interest in \gls{snare} drug delivery mechanisms.

\gls{snare}s can fuse liposomes containing various drug payloads into the membrane.
Notably, known \gls{snare} \gls{tmh}s are highly hydrophobic even compared to other tail anchored \gls{tmh}s~\cite{Kalbfleisch2007}.
This hydrophobicity appears to be a determinate factor in the precise delivery mechanistic route that a~\gls{ta} proteins use for insertion~\cite{Rabu2008, Rabu2009}, for which there is evidence demonstrating that are several mechanisms~\cite{Rabu2009, Johnson2013}.

Whilst most eukaryotic~\gls{ta} proteins are inserted into the~\gls{er}.

Their studies have moved the focus of spontaneous insertion away from the loop regions and onto the physicochemical factors of the \gls{tmh} itself.

\section{Methods}

\subsection{Building a List of Tail-Anchors}
Steps carried out by Kalbfleisch \textit{et al.} published in Traffic 2007 (8: 1687\-1694)~\cite{Kalbfleisch2007}, were recreated using up to date tools.
Whilst their study focused on the human proteome, here we take into account the entire TrEMBL and SwissProt database and then stratify the datasets by the organism at the end of the pipeline.

\subsubsection{SwissProt Tail Anchored Dataset According to Filters}
There were 557012 protein records downloaded from SwissProt via UniProt~\cite{TheUniProtConsortium2014} (Downloaded 24--04--2018).
106149~\gls{tmh}s (\url{TRANSMEM} annotation) were found between 76953 records (\url{annotation:(type:transmem) AND reviewed:no}).
This keyword is contained in a record according to either experimental evidence~\cite{TheUniProtConsortium2014} or a robust meta-analysis of~\gls{tmh} prediction using TMHMM~\cite{Krogh2001}, Memsat~\cite{Jones2007}, Phobius~\cite{Kall2004,Kall2007} and the hydrophobic moment plot method of Eisenberg and co-workers~\cite{Eisenberg1984}.
11141 of those records had only a single~\gls{tmh}.
11110 of those~\gls{tmh}s were within the length thresholds of 16 to 30 residues (None of those had the annotation for splice isoforms according to \url{NON_TER} annotation).
5548 of those had had no~\gls{sp} annotation (\url{SIGNAL}).
4332 of those had annotation (based on \url{TOPO_DOM} annotation) that the N terminal was cytoplasmic.
615 of those had the~\gls{tmh} within 25 residues of the C terminal, the same threshold used by Kalbfleisch and their coworkers~\cite{Kalbfleisch2007}.
Running CD-Hit 4.5.3 on the WebMGA web-server~\cite{Huang2010, Wu2011} at 90\% identical sequence at 90\% coverage thresholds resulted in 443 representative proteins.
This threshold was chosen as a compromise between avoiding over-representation of a certain protein and maintaining a viable sample size.

From this representative list, 46 were Archaeal, 66 were bacterial, and 320 were Eukaryotic and 11 came from dsDNA viruses.
When counting proteomes with greater than 20 records, 49 belonged to the \textit{A. thaliana} proteome, 48 to Mouse, 46 to the human proteome, 24 to \textit{S.cerevisiae}. %19 from RAT!

65 were annotated under the Mitochondrion location (query \url{locations:(location:"Mitochondrion [SL-0173]")}), 157 in the \gls{pm} (query \url{locations:(location:"Cell membrane [SL-0039]")}, 82 in the Golgi (query \url{locations:(location:"Golgi apparatus [SL-0132]")}), and 98 in the \gls{er} (query \url{locations:(location:"Endoplasmic reticulum [SL-0095]"}).

\subsubsection{TrEMBL Tail Anchored Dataset According to Filters}
111425234 records were stored in the TrEMBL database at time of download (Downloaded 25--04--2018).
22107826 of those contained \url{TRANSMEM} annotation (\url{annotation:(type:transmem) AND reviewed:no}).
18053 of these were single-pass proteins.
All of these were within the length restrictions.
17973 of those did not contain a signal sequence when looking for \url{SIGNAL} annotation.
5157 of those contained a cytoplasmically located N terminal according to \url{TOPO_DOM} annotation.
155 records had a~\gls{tmh} within 15 residues of the C terminal residue.
In those record's annotations, no more than 1 appeared in any given species, so they were omitted from the SwissProt list sequence redundancy protocol to avoid representing a well-annotated record with a poorly annotated record.

\subsubsection{UniProt Curated List}
A query for \url{locations:(location:"Single-pass type IV membrane protein [SL-9908]")} was used in UniProt which returned 2460 UniProtKB IDs; 463 SwissProt results and 1997 TrEMBL results.
Running these records through CD-HIT at 90\% redundancy yielded 309 SwissProt records and 808 TrEMBL records~\cite{Huang2010, Wu2011}.
Of those, 987 proteins from 973 records (308 from SwissProt, and 665 from TrEMBL) had the \url{TRANSMEM} annotation indicating a bone fide~\gls{tmh}.
No further filters were applied to this list.
Proteomes represented by more than 20 records include \textit{A. thaliana} (60 records), Humans (38), Mouse (37), and \textit{S. cerevisiae} (31). % and 20 to Rat.

401 were annotated under the Mitochondrion location (query \url{locations:(location:"Mitochondrion [SL-0173]")}) 39 from SwissProt and 362 automatically assigned in TrEMBL.
401 in the \gls{er} (query \url{locations:(location:"Endoplasmic reticulum [SL-0095]"}), 98 from SwissProt and 303 automatically annotated in TrEMBL.
1 TrEMBL record (A0A1E5RT24) in the \gls{er} set contained an ``X'' residue in the C terminal flank and was omitted from the analyses.
Two subcellular location datasets had no automatically ascribed records and only contained manually annotated SwissProt records; 37 in the \gls{pm} (query \url{locations:(location:"Cell membrane [SL-0039]")}, and 82 in the Golgi (query \url{locations:(location:"Golgi apparatus [SL-0132]")}).

\subsubsection{Remapping Previous Dataset}
189 of the 411 proteins from the previous study~\cite{Kalbfleisch2007} were successfully mapped to 222 UniProtKB IDs using the UniProt mapping tools with the RefSeq Protein to UniProtKB option~\cite{TheUniProtConsortium2014}.

\subsection{Calculating Hydrophobicity}
Windowed hydrophobicity was calculated using a window length of 5 residues, and half windows were permitted.
Average hydrophobicity takes the total of the raw amino acid hydrophobicity values and divides them by the number of amino acids in the slice.
Values reported in the results are based on the Kyte \& Doolittle scale~\cite{Kyte1982} which is based on the water\---vapour transfer free energy and the interior-exterior distribution of individual amino acids.
%Hydrophobicity values were also validated by the White and Wimley scale~\cite{White1999}, the Hessa scale~\cite{Hessa2005}, and the Eisenberg scale~\cite{Eisenberg1984}.

\subsection{Calculating Sequence Information Entropy}
Information entropy, is essentially an estimate of the linguistic entropy of a string.
In the context of biology, it can be thought of as an estimation of the non-randomness of a sequence.
Sequence complexity can be used to analyse DNA sequences~\cite{Pinho2013, Oliver1993, Troyanskaya2002}, and is a component of the TMSOC z-score which can predict function beyond anchorage of a \gls{tmh}~\cite{Wong2011, Wong2012, Baker2017}.
here we focus on the analysis of the complexity of a sequence in protein sequences.

Broadly speaking, the information theory entropy of a linguistic string can be defined as in equation~\ref{simpleentropy2}, and we treat the protein sequence \gls{tmh} as a string with or without its flanking regions.

\begin{equation} \label{simpleentropy2}
	H(S)=-{\sum_{i=1}^n {p_i\log_s(p_i)}}
\end{equation}

Where H is the entropy of a sequence (S), and $p_i$ is the probability of a character $i$ through each position (n) in S. This allows us to quantify the average relative information density held within a string of information~\cite{Shannon1948}.

\subsection{Statistics}

The null hypothesis of homogeneity of two distributions was examined with the Kolmogorov Smirnov, the Kruskal-Wallis, and the 2-sampled Student's T-test statistical tests.
These tests were all ran through the Python SciPy stat package v0.17 python package~\cite{VanderWalt2011}.
To note, the~\gls{ks} test scrutinises for significant maximal absolute differences between distribution curves; the~\gls{kw} test is after skews between distributions and the student t-test statistical test checks the average difference between distributions.

Since the P‑value is a product of a fraction of test statistics obtained from a permutated set of the samples, it exponentially increases as N increases; the P-value is a strong function of N.
We rely on the (absolute) Bahadur slope ($B$) as a measure of distance between two distributions~\cite{Bahadur1967, Bahadur1971, Sunyaev1998, Baker2017}:

\begin{equation} \label{eq:bahadur2}
B=\frac{|\ln(P~value)|}{N}
\end{equation}

The larger the absolute Bahadur slope, the greater the difference between the two distributions.


\section{Results}

\subsection{A Comparison Of Up-To-Date Tail-Anchored Protein Datasets}
Here, we use two sources for \gls{ta} protein datasets.
One dataset is based on a previous method~\cite{Kalbfleisch2007} to obtain \gls{ta} datasets and consists of 9296 \gls{tmh} residues (13279 including up to $\pm$5 flanking residues) from 443 SwissProt entries with 90\% redundancy removal.
Another dataset contains the UniProt curated set of Type IV membrane proteins again with 90\% redundancy removal.
This dataset contains 21119 \gls{tmh} residues (28791 including up to $\pm$5 flanking residues) from 987 UniProt protein records.

% Venn diagram
\begin{figure}[!ht]
\centering
\includegraphics[width=0.5\textwidth]{TA_chapter/database-overlap}
		\captionof{figure}[A Venn diagram showing tail anchored protein UniProt ids present in each of the datasets as well as those present in multiple datasets.]{\textbf{A Venn diagram showing tail anchored protein UniProt ids present in each of the datasets as well as those present in multiple datasets.}
The number of ids present in redundant versions of
i) the supplementary materials table of a previous study predicting the complete set of human tail anchored proteins denote by S1~\cite{Kalbfleisch2007},
ii) the SwissProt dataset filtered according to typical~\gls{ta} features limited to the human proteome~\cite{TheUniProtConsortium2014}, and
iii) The UniProt curated list of~\gls{ta} proteins~\cite{TheUniProtConsortium2014}.
Note that to avoid losing IDs to redundancy reduction this diagram was generated without the use of CD-HIT~\cite{Huang2010, Wu2011}, which is applied in later statistical analysis.}

\label{fig:tadatasetoverlap}
\end{figure}

In order to get an understanding of the consistency of the datasets, before removing redundant proteins, we compared these two datasets to a dataset remapped set of proteins from a previous 2007 method~\cite{Kalbfleisch2007}.
The S1 dataset was built with an aim to gather \gls{ta} proteins in the human genome from the NCBI.
Note that these numbers are not absolutely certain.
The greatest source of uncertainty here is that the original S1 list includes 411 records, however only 222 of these were successfully mapped to the UniProt dataset.
This figure is closer to the 202 proteins from the original S1 list that excluded proteins that were either hypothetical or splice isoforms.
That being said, this mapping step prevents us from directly comparing the entire original S1 dataset.
We compared the up-to-date datasets to S1 to see how many records are shared, how many are now obsolete, and how many are unique.

Figure~\ref{fig:tadatasetoverlap} shows that S1 has 175 record ids of 222 records (78.8\%) which do not share overlap the up-to-date manually curated UniProt dataset~\cite{TheUniProtConsortium2014}.
Of the 166 unique records of that S1 dataset, 92 records do have location annotation in UniProt which our scripts use for topological determination, leaving 74 records without location annotation.
This leaves 92 of the 222 S1 (41.4\%) records that originally fitted criteria that no longer fit those same criteria.
If we exclude those lacking suitable annotation i.e ids from S1 that are found in either SwissProt with the filters (9), the curated UniProt list (14), or both (33), compared to the 92 that have annotation contradicting the original predictions, 37.8\% of the ids overlap.

Equivalent criteria were applied to the entire SwissProt database and then restricted to the human proteome dataset.
43 of these 77 records (55.8\%) are in the curated UniProt~\gls{ta}  dataset leaving 34 records that meet the criteria out of the manually curated set (44.2\%of the filtered SwissProt dataset).
42 of the 77 (54.5\%) records from SwissProt filtered human dataset can be found in the original S1 list.
A further consideration is that, before limiting the dataset to human and after removing redundant proteins, this method picked up  46 archaeal and 66 bacterial records.

The same method applied to an up-to-date dataset overlaps more with a manually curated dataset.
There is also a large degree of what we now believe to be mistakes that occurred in the older prediction tools and datasets, even when using similar methods.
As a trend, this shows that up-to-date datasets improve the reliability of this automated predicted method.
These automated criteria still do not fully align with the manually curated list.
Only 973 of a non-redundant (90\% CD-HIT threshold \cite{Huang2010, Wu2011}) set of those 2460 proteins of the UniProt manually curated set contained annotation for the transmembrane boundary residues.
Ultimately, this points to the idea that datasets are a moving target as they are constantly updated with more accurate and reliable tools.

\subsection{It Is Difficult To Observe Hydrophobic Variation Of TA Protein TMHs From Different Species}

In single-pass proteins of eukaryotic species there are typically various adaptations of the \gls{tmh} to adhere to the membrane constraints of the specific membrane.
For single-pass proteins, previous studies have observed differences in terms of \gls{tmh} hydrophobicity between yeast and human \gls{tmp}s~\cite{Sharpe2010}, or in cress, yeast, bacteria, and human datasets~\cite{Baker2017}.
We would expect to see a similar trend between the \gls{tmh}s of \gls{ta} proteins from different species.
However, when assuming a zero-difference hypothesis, in these \gls{tmh} \gls{ta} protein datasets we cannot observe any species-level differences between the datasets at this sample size for \gls{tmh} hydrophobicity.

% Average lines for figure, table for stats
When comparing the average Kyte \& Doolittle~\cite{Kyte1982} hydrophobicity values for the~\gls{tmh}s from humans and mice, \textit{A. thaliana}, and  \textit{S. cerevisiae}, we can see little difference between the mean values.
All of the mean values lie between 2.3-2.6 when we only consider the \gls{tmh} and at 1.3-1.6 when considering residues in close proximity to the~\gls{tmh} ($\pm$5 residues) (Figure~\ref{fig:average_species_hydrophobicity_ta}).

\begin{figure}[!ht]
\centering
\includegraphics[width=1\textwidth]{TA_chapter/species-hydrophobicity}
\captionof{figure}[Average values of species datasets from UniProt manually curated set and SwissProt automatically filtered dataset.]
{\textbf{Average values of species datasets from UniProt manually curated set and SwissProt automatically filtered dataset.}

The average hydrophobicity values from the Kyte \& Doolittle scale~\cite{Kyte1982}.for both the \gls{tmh} and the \gls{tmh}$\pm$5 residues.
%and the GlobProt scale~\cite{Linding2003}
Values are shown for both the UniProt manually curated set and the SwissProt filtered set. In the UniProt manually curated set we compare the mammalian set of \gls{ta} proteins (Human N=38 and Mouse N=37) to \textit{A. thaliana} (N=60) representing plants and \textit{S. cerevisiae} (N=31) representing yeasts. For the SwissProt filtered set we compare the mammalian set of \gls{ta} proteins (Human N=46 and Mouse N=48) to \textit{A. thaliana} (N=49) representing plants  and  \textit{S. cerevisiae} (N=24) representing yeasts.
Error bars are shown at $\pm 1 \sigma$ from the mean of the respective dataset.
}

\label{fig:average_species_hydrophobicity_ta}
\end{figure}

Indeed, we see no strong observable statistical differences in hydrophobicity ($P>3.35E-1$ in the SwissProt automatically filtered list Table \ref{table:speciestableswissprotstats}, and $P>8.30E-2$ in the UniProt curated list Table \ref{table:speciestableuniprotstats}).
There are also no consistent trends among the absolute Bahadur slopes; no datasets are greatly different from any other.

\begin{table}[htbp]
\centering
\captionof{table}[Hydrophobicity statistical comparisons between mouse and human, yeast, and plants in the SwissProt Filtered Dataset.]
{\textbf{Hydrophobicity statistical comparisons between mouse and human, yeast, and plants in the SwissProt Filtered Dataset.}
Here, we compare a mammalian set of \gls{ta} proteins (Human N=46 and Mouse N=48) to \textit{A. thaliana} (N=49) representing plants  and  \textit{S. cerevisiae} (N=24) representing yeasts.
The hydrophobicity was predicted as the mean average of the values of the sequences of the \gls{tmh}, as well another group including up to $\pm$5 flanking residues, since predicting the boundary of \gls{tmh}s is difficult, according to the Kyte \& Doolittle hydrophobicity scale~\cite{Kyte1982}.
% Disorder was calculated in the same way using the GlobProt scale \cite{Linding2003}.
The Test column refers to the statistical score obtained from the test; H statistic for the Kruskal Wallis, the KS statistic for the Kolmogorov Smirnov test, and the t-statistic for the T-test.
$P$ is the P-value of that statistical score.
$B$ refers to the Bahadur slope, an interpretation of the P-value that accounts for the sample size powering the test~\cite{Bahadur1967, Bahadur1971}.}
\tiny
	% Table generated by Excel2LaTeX from sheet 'SwissProt filtered species'

    \begin{tabular}{clrrrrrrrrr}
          &       & \multicolumn{3}{c}{Mammal and Plant} & \multicolumn{3}{c}{Mammal and Yeast} & \multicolumn{3}{c}{Plant and Yeast} \\
          &       & \multicolumn{1}{l}{Test} & \multicolumn{1}{l}{$P$} & \multicolumn{1}{l}{$B$} & \multicolumn{1}{l}{Test} & \multicolumn{1}{l}{$P$} & \multicolumn{1}{l}{$B$} & \multicolumn{1}{l}{Test} & \multicolumn{1}{l}{$P$} & \multicolumn{1}{l}{$B$} \\
    \multirow{3}[0]{*}{TMH } &  Kruskal-Wallis & 0.93  & 3.35E-1 & 7.64E-3 & 0.10  & 7.56E-1 & 2.37E-3 & 0.84  & 3.60E-1 & 1.40E-2 \\
          &  Kolmogorov-Smirnov & 0.13  & 6.36E-1 & 3.17E-3 & 0.12  & 9.24E-1 & 6.69E-4 & 0.19  & 5.28E-1 & 8.76E-3 \\
          &  Student's T-test & -0.86 & 3.90E-1 & 6.58E-3 & 0.21  & 8.31E-1 & 1.57E-3 & 0.79  & 4.33E-1 & 1.15E-2 \\
    \multirow{3}[0]{*}{TMH and flanks } &  Kruskal-Wallis & 0.04  & 8.52E-1 & 1.12E-3 & 0.12  & 7.28E-1 & 2.69E-3 & 0.04  & 8.33E-1 & 2.51E-3 \\
          &  Kolmogorov-Smirnov & 0.11  & 7.72E-1 & 1.81E-3 & 0.13  & 8.79E-1 & 1.09E-3 & 0.11  & 9.80E-1 & 2.81E-4 \\
          &  Student's T-test & -0.22 & 8.23E-1 & 1.37E-3 & -0.38 & 7.04E-1 & 2.97E-3 & -0.19 & 8.50E-1 & 2.22E-3 \\
    \end{tabular}%
				\label{table:speciestableswissprotstats}

\end{table}%

\begin{table}[htbp]
\centering
\captionof{table}[Hydrophobicity statistical comparisons between mouse and human, yeast, and plants in the UniProt Curated Dataset.]
{\textbf{Hydrophobicity statistical comparisons between mouse and human, yeast, and plants in the UniProt Curated Dataset.}
Here, we compare a mammalian set of \gls{ta} proteins (Human N=38 and Mouse N=37) to \textit{A. thaliana} (N=60) representing plants  and  \textit{S. cerevisiae} (N=31) representing yeasts.
The hydrophobicity was predicted as the mean average of the values of the sequences of the \gls{tmh}, as well another group including up to $\pm$5 flanking residues, since predicting the boundary of \gls{tmh}s is difficult, according to the Kyte \& Doolittle hydrophobicity scale~\cite{Kyte1982}.
%Disorder was calculated in the same way using the GlobProt scale \cite{Linding2003}.
The Test column refers to the statistical score obtained from the test; H statistic for the Kruskal Wallis, the KS statistic for the Kolmogorov Smirnov test, and the t-statistic for the T-test.
$P$ is the P-value of that statistical score.
$B$ refers to the Bahadur slope, an interpretation of the P-value that accounts for the sample size powering the test~\cite{Bahadur1967, Bahadur1971}.}
	\tiny
	% Table generated by Excel2LaTeX from sheet 'SwissProt filtered species'

    \begin{tabular}{clrrrrrrrrr}
	          &       & \multicolumn{3}{c}{Mammal and Plant} & \multicolumn{3}{c}{Mammal and Yeast} & \multicolumn{3}{c}{Plant and Yeast} \\
	          &       & \multicolumn{1}{l}{Test} & \multicolumn{1}{l}{$P$} & \multicolumn{1}{l}{$B$} & \multicolumn{1}{l}{Test} & \multicolumn{1}{l}{$P$} & \multicolumn{1}{l}{$B$} & \multicolumn{1}{l}{Test} & \multicolumn{1}{l}{$P$} & \multicolumn{1}{l}{$B$} \\
	    \multirow{3}[0]{*}{TMH } &  Kruskal-Wallis & 2.15  & 1.42E-1 & 1.45E-2 & 0.22  & 6.39E-1 & 4.22E-3 & 2.30  & 1.30E-1 & 2.25E-2 \\
	          &  Kolmogorov-Smirnov & 0.17  & 2.86E-1 & 9.27E-3 & 0.15  & 6.32E-1 & 4.34E-3 & 0.24  & 1.69E-1 & 1.95E-2 \\
	          &  Student's T-test & -1.71 & 8.96E-2 & 1.79E-2 & 0.04  & 9.70E-1 & 2.86E-4 & 1.47  & 1.46E-1 & 2.11E-2 \\
	    \multirow{3}[0]{*}{TMH and flanks } &  Kruskal-Wallis & 2.17  & 1.41E-1 & 1.45E-2 & 0.14  & 7.13E-1 & 3.19E-3 & 0.59  & 4.41E-1 & 9.00E-3 \\
	          &  Kolmogorov-Smirnov & 0.21  & 8.30E-2 & 1.84E-2 & 0.10  & 9.62E-1 & 3.62E-4 & 0.14  & 8.00E-1 & 2.45E-3 \\
	          &  Student's T-test & -1.33 & 1.86E-1 & 1.25E-2 & -0.39 & 7.00E-1 & 3.36E-3 & 0.69  & 4.90E-1 & 7.83E-3 \\
	    \end{tabular}%
					\label{table:speciestableuniprotstats}
	\end{table}%

Here, we are dealing with datasets at least an order of magnitude smaller than those broad studies \cite{Sharpe2010, Baker2017} which could explain the absence of the effect.
However this only goes to show that if there is an effect in \gls{ta} proteins, it is indeed weak between species.


\subsection{There Are Biochemical Differences Between Tail-Anchored TMHs From Different Organelles}
% Average lines for figure, table for stats

As in the case of species, \gls{tmh}s with different subcellular localisations on average  have different hydrophobicity.
%Need mitochondria reference, and numbers for the UniER etc.
These hydrophobic differences have already been observed in the mitochondria localised \gls{ta} protein \gls{tmh}.
Here we consider the \gls{ta} proteins at certain locations within the cell ignoring species, and we see clear differences in the biochemistry of the \gls{tmh}.
In the UniProt manually curated dataset, the Kyte \& Doolittle hydrophobicity scores rage from 1.7 in mitochondria to 2.7 in the \gls{pm} (Figure \ref{fig:average_organelle_factors_ta}A).

\begin{figure}[!ht]
\centering
\includegraphics[width=1\textwidth]{TA_chapter/organelle-averages}
\captionof{figure}[Average sequence-based biochemical values of organelle datasets from UniProt manually curated set and SwissProt automatically filtered dataset.]
{\textbf{Average sequence-based biochemical values of organelle datasets from UniProt manually curated set and SwissProt automatically filtered dataset.}

A) The average hydrophobicity values from the Kyte \& Doolittle scale~\cite{Kyte1982}, B) the average information entropy~\cite{Shannon1948} (see methods) and C) the GlobProt scale~\cite{Linding2003} for both the \gls{tmh} and the \gls{tmh}$\pm$5 residues.
Values are shown for both the UniProt manually curated set and the SwissProt filtered set.
In the UniProt manually curated set we compare \gls{ta} proteins from the~\gls{er} (N=400) to the Golgi (N=82), the~\gls{pm} (N=37), and the mitochondria (N=401).
For the SwissProt filtered set we compare \gls{ta} proteins from the~\gls{er} (N=98) to the Golgi (N=82), the~\gls{pm} (N=157), and the mitochondria (N=65).
Error bars are shown at $\pm 1 \sigma$ from the mean of the respective dataset.
}

\label{fig:average_organelle_factors_ta}
\end{figure}




\begin{table}[htbp]
\centering
\captionof{table}[Statistical comparisons between TMH sequences from organelles in the UniProt Curated Dataset.]
{\textbf{Statistical comparisons between TMH sequences from organelles in the UniProt Curated Dataset.}
Here, we compare a organelle subsets from the UniProt curated dataset of \gls{ta} proteins.
We compare \gls{er} (N=400) to Golgi (N=82), \gls{pm} (N=37), and the mitochondria (N=401).
The hydrophobicity was predicted as the mean average of the values of the sequences of the \gls{tmh}, as well another group including up to $\pm$5 flanking residues, since predicting the boundary of \gls{tmh}s is difficult, according to the Kyte \& Doolittle hydrophobicity scale~\cite{Kyte1982}.
Disorder was calculated in the same way using the GlobProt scale \cite{Linding2003}.
The linguistic information entropy was calculated according to the methods section~\cite{Shannon1948}.
The Test column refers to the statistical score obtained from the test; H statistic for the Kruskal Wallis (KW), the KS statistic for the Kolmogorov Smirnov test (KS), and the t-statistic for the student's T-test (T-test).
$P$ is the P-value of that statistical score.
$B$ refers to the Bahadur slope, an interpretation of the P-value that accounts for the sample size powering the test~\cite{Bahadur1967, Bahadur1971}.}
	\tiny

	\begin{tabular}{clrrrrrrrrr}
							&       & \multicolumn{3}{c}{ER and Golgi} & \multicolumn{3}{c}{ER and PM} & \multicolumn{3}{c}{ER and mitochondria} \\
							&       & \multicolumn{1}{l}{Test} & \multicolumn{1}{l}{$P$} & \multicolumn{1}{l}{$B$} & \multicolumn{1}{l}{Test} & \multicolumn{1}{l}{$P$} & \multicolumn{1}{l}{$B$} & \multicolumn{1}{l}{Test} & \multicolumn{1}{l}{$P$} & \multicolumn{1}{l}{$B$} \\
	\midrule
	\multirow{3}[0]{*}{TMH Hydrophobicity} &  KW & 15.84 & 6.89E-5 & 1.93E-2 & 32.82 & 1.01E-8 & 4.17E-2 & 344.94 & 5.36E-77 & 2.15E-1 \\
							&  KS & 0.30  & 2.79E-6 & 2.58E-2 & 0.54  & 2.04E-9 & 4.54E-2 & 0.64  & 3.71E-74 & 2.07E-1 \\
							&  T-test & -5.45 & 7.91E-8 & 3.30E-2 & -8.64 & 1.02E-16 & 8.35E-2 & 21.65 & 1.90E-82 & 2.31E-1 \\
	\midrule
	\multirow{3}[0]{*}{... including flanks} &  KW & 0.95  & 3.30E-1 & 2.24E-3 & 15.29 & 9.22E-5 & 2.11E-2 & 463.33 & 9.05E-103 & 2.88E-1 \\
							&  KS & 0.17  & 1.79E-2 & 8.11E-3 & 0.47  & 3.90E-7 & 3.35E-2 & 0.80  & 1.03E-116 & 3.28E-1 \\
							&  T-test & 0.52  & 6.02E-1 & 1.02E-3 & -4.72 & 3.24E-6 & 2.87E-2 & 32.76 & 1.02E-150 & 4.24E-1 \\
	\midrule
	\multirow{3}[0]{*}{TMH Disorder} &  KW & 79.40 & 5.08E-19 & 8.49E-2 & 24.33 & 8.12E-7 & 3.18E-2 & 60.00 & 9.46E-15 & 3.96E-2 \\
							&  KS & 0.48  & 5.56E-16 & 7.08E-2 & 0.49  & 8.84E-8 & 3.68E-2 & 0.26  & 3.78E-12 & 3.23E-2 \\
							&  T-test & 10.62 & 7.27E-24 & 1.07E-1 & 6.03  & 3.43E-9 & 4.42E-2 & -7.98 & 4.81E-15 & 4.05E-2 \\
	\midrule
	\multirow{3}[0]{*}{... including flanks} &  KW & 74.45 & 6.22E-18 & 7.99E-2 & 13.54 & 2.33E-4 & 1.90E-2 & 33.76 & 6.23E-9 & 2.32E-2 \\
							&  KS & 0.46  & 1.58E-14 & 6.41E-2 & 0.37  & 1.42E-4 & 2.01E-2 & 0.22  & 3.02E-9 & 2.41E-2 \\
							&  T-test & 9.93  & 2.52E-21 & 9.56E-2 & 3.78  & 1.78E-4 & 1.96E-2 & -5.96 & 3.77E-9 & 2.38E-2 \\
	\midrule
	\multirow{3}[0]{*}{TMH entropy} &  KW & 1.33  & 2.49E-1 & 2.80E-3 & 31.76 & 1.75E-8 & 4.05E-2 & 22.64 & 1.95E-6 & 1.61E-2 \\
							&  KS & 0.20  & 3.86E-3 & 1.12E-2 & 0.42  & 6.23E-6 & 2.72E-2 & 0.18  & 1.46E-6 & 1.65E-2 \\
							&  T-test & 1.97  & 4.97E-2 & 6.05E-3 & 7.37  & 8.47E-13 & 6.30E-2 & -4.24 & 2.54E-5 & 1.30E-2 \\
	\midrule
	\multirow{3}[0]{*}{... including flanks} &  KW & 0.16  & 6.93E-1 & 7.39E-4 & 27.12 & 1.91E-7 & 3.51E-2 & 42.77 & 6.15E-11 & 2.88E-2 \\
							&  KS & 0.16  & 3.15E-2 & 6.97E-3 & 0.46  & 5.92E-7 & 3.25E-2 & 0.24  & 1.20E-10 & 2.80E-2 \\
							&  T-test & 0.10  & 9.21E-1 & 1.66E-4 & 5.55  & 4.97E-8 & 3.81E-2 & -6.14 & 1.26E-9 & 2.51E-2 \\
	\end{tabular}%
					\label{table:organellesuniprotstats}
	\end{table}%

In the UniProt curated list, there are clear hydrophobic differences between all the \gls{tmh}s ($P<6.89E-5$) which as a trend becomes less clear when considering the \gls{tmh}$\pm$5 flanking residues except for mitochondria which increases in significance when considering the flanks also (Table~\ref{table:organellesuniprotstats}).
The \gls{er} and mitochondrial tests are very significant ($P<3.71E-74$).
Consistently the Bahadur slope is at least an order of magnitude greater in the \gls{er} and mitochondrial comparison than for the other considerations, so these differences cannot be accounted for by the larger sample size.

In terms of information entropy, there is a marked decrease in entropy in the \gls{pm} subset (mean entropy = 3.12 in the \gls{tmh}, 2.6 including $\pm$5 flanking residues) from the UniProt curated dataset compared to the other organelle datasets (entropy $>$ 3.3 and $>$ 2.8 including the flanks).
However this stark difference between \gls{tmh}s from \gls{pm} bound \gls{ta} proteins and the other organelle datasets cannot be observed in the SwissProt set (Figure~\ref{fig:average_organelle_factors_ta}).
All considerations of disorder were significant ($P<2.33E-4$)(Table~\ref{table:organellesuniprotstats}).
As a trend, the flanks obscure the differences between the datasets, implying the differences of disorder are more clear within the \gls{tmh} itself.

No clear significances can be observed for the information entropy ($P>6.33E-2$).
This is unsurprising given that the hydrophobic nature of the \gls{tmh}s demands that certain residues must be over-represented, which lowers the information entropy.
In this case, we have a highly hydrophobic set, the \gls{pm} UniProt set, which likely contains a higher proportion of the most hydrophobic residues.
As a trend the entropy mirrors the hydrophobicity albeit with less range between dataset means (2.62-3.00 in the \gls{tmh} for disorder, 1.72-2.74 for hydrophobicity)(Figure~\ref{fig:average_organelle_factors_ta}).


	\begin{table}[htbp]
	\centering
	\captionof{table}[Statistical comparisons between TMH sequences from organelles in the SwissProt Filtered Dataset.]
	{\textbf{Statistical comparisons between TMH sequences from organelles in the SwissProt Filtered Dataset.}
	Here, we compare a organelle subsets from the SwissProt automatically filtered dataset of \gls{ta} proteins.
	We compare \gls{er} (N=98) to Golgi (N=82), \gls{pm} (N=157), and the mitochondria (N=65).
	The hydrophobicity was predicted as the mean average of the values of the sequences of the \gls{tmh}, as well another group including up to $\pm$5 flanking residues, since predicting the boundary of \gls{tmh}s is difficult, according to the Kyte \& Doolittle hydrophobicity scale~\cite{Kyte1982}.
	Disorder was calculated in the same way using the GlobProt scale \cite{Linding2003}.
	The linguistic information entropy was calculated according to the methods section~\cite{Shannon1948}.
	The Test column refers to the statistical score obtained from the test; H statistic for the Kruskal Wallis (KW), the KS statistic for the Kolmogorov Smirnov test (KS), and the t-statistic for the student's T-test (T-test).
	$P$ is the P-value of that statistical score.
	$B$ refers to the Bahadur slope, an interpretation of the P-value that accounts for the sample size powering the test~\cite{Bahadur1967, Bahadur1971}.}
		\tiny
		% Table generated by Excel2LaTeX from sheet 'SwissProt filtered species'

		%\begin{tabular}{clrrrrrrrrr}
		 \begin{tabular}{clccccccccc}
								&       & \multicolumn{3}{c}{ER and Golgi} & \multicolumn{3}{c}{ER and PM} & \multicolumn{1}{l}{ER and mitochondria} &       &  \\
								&       & \multicolumn{1}{l}{Test} & \multicolumn{1}{l}{$P$} & \multicolumn{1}{l}{$B$} & \multicolumn{1}{l}{Test} & \multicolumn{1}{l}{$P$} & \multicolumn{1}{l}{$B$} & \multicolumn{1}{l}{Test} & \multicolumn{1}{l}{$P$} & \multicolumn{1}{l}{$B$} \\

		\midrule
		\multirow{3}[0]{*}{TMH Hydrophobicity} &  KW & 11.96 & 5.43E-4 & 4.18E-2 & 0.02  & 8.77E-1 & 5.14E-4 & 8.46  & 3.64E-3 & 3.45E-2 \\
								&  KS & 0.27  & 1.98E-3 & 3.46E-2 & 0.08  & 8.48E-1 & 6.44E-4 & 0.27  & 4.62E-3 & 3.30E-2 \\
								&  T-test & -3.47 & 6.50E-4 & 4.08E-2 & -0.17 & 8.67E-1 & 5.60E-4 & 3.45  & 7.24E-4 & 4.44E-2 \\
		\midrule
		\multirow{3}[0]{*}{... including flanks} &  KW & 5.92  & 1.50E-2 & 2.33E-2 & 9.14  & 2.50E-3 & 2.35E-2 & 26.42 & 2.75E-7 & 9.27E-2 \\
								&  KS & 0.21  & 2.85E-2 & 1.98E-2 & 0.26  & 4.88E-4 & 2.99E-2 & 0.43  & 4.93E-7 & 8.91E-2 \\
								&  T-test & -2.52 & 1.25E-2 & 2.43E-2 & -3.09 & 2.23E-3 & 2.40E-2 & 4.95  & 1.87E-6 & 8.09E-2 \\
	  \midrule
		\multirow{3}[0]{*}{TMH Disorder} &  KW & 18.84 & 1.42E-5 & 6.20E-2 & 0.37  & 5.44E-1 & 2.39E-3 & 28.06 & 1.17E-7 & 9.79E-2 \\
								&  KS & 0.30  & 5.01E-4 & 4.22E-2 & 0.12  & 3.15E-1 & 4.53E-3 & 0.41 & 2.87E-6 & 7.83E-2 \\
								&  T-test & 4.74  & 4.34E-6 & 6.86E-2 & -0.33 & 7.40E-1 & 1.18E-3 & -5.33 & 3.22E-7 & 9.17E-2 \\
		\midrule
		\multirow{3}[0]{*}{... including flanks} &  KW & 15.13 & 1.01E-4 & 5.11E-2 & 3.26  & 7.08E-2 & 1.04E-2 & 29.19 & 6.57E-8 & 1.01E-1 \\
								&  KS & 0.28  & 1.37E-3 & 3.66E-2 & 0.16  & 8.69E-2 & 9.58E-3 & 0.43 & 4.88E-7 & 8.92E-2 \\
								&  T-test & 4.34  & 2.36E-5 & 5.92E-2 & -1.53 & 1.27E-1 & 8.10E-3 & -5.23 & 5.16E-7 & 8.88E-2 \\
		\midrule
		\multirow{3}[0]{*}{TMH entropy} &  KW & 2.96  & 8.56E-2 & 1.37E-2 & 0.66  & 4.17E-1 & 3.43E-3 & 0.69  & 4.05E-1 & 5.54E-3 \\
								&  KS & 0.13  & 4.32E-1 & 4.66E-3 & 0.10  & 5.27E-1 & 2.51E-3 & 0.18 & 1.40E-1 & 1.20E-2 \\
								&  T-test & 1.58  & 1.15E-1 & 1.20E-2 & 0.79  & 4.32E-1 & 3.29E-3 & 1.03 & 3.06E-1 & 7.26E-3 \\
		\midrule
		\multirow{3}[0]{*}{... including flanks} &  KW & 2.62  & 1.06E-1 & 1.25E-2 & 2.87  & 9.04E-2 & 9.42E-3 & 0.05 & 8.31E-1 & 1.14E-3 \\
								&  KS & 0.15  & 2.48E-1 & 7.75E-3 & 0.17  & 6.56E-2 & 1.07E-2 & 0.21 & 6.33E-2 & 1.69E-2 \\
								&  T-test & 1.84  & 6.75E-2 & 1.50E-2 & 1.66  & 9.84E-2 & 9.09E-3 & 0.42 & 6.72E-1 & 2.44E-3 \\
		\end{tabular}%
						\label{table:organellesswissstats}
		\end{table}%

Similarly, in the SwissProt filtered dataset the mean \gls{tmh} hydrophobicity for mitochondria is the lowest at 1.9, but it appears to be the Golgi apparatus that is the peak at 2.4.
In the SwissProt dataset, when we compare each subset of only the \gls{tmh} to the \gls{er} subset, we find significance between the \gls{er} and the Golgi ($P<1.98E-3$), and the \gls{er} and the mitochondria ($P<4.62E-3$), however the \gls{er} and \gls{pm} are more similar considering the Bahadur values are $<6.44E-4$, two orders of magnitude smaller that the other sets (Bahadur values $>3.3E-2$) (Table \ref{table:organellesswissstats}).
When we take into account the flanks, the \gls{er} and \gls{pm} dataset can be distinguished ($P<2.50E-3$), however as a trend the other two comparisons, \gls{er} and Golgi becomes less significant, and \gls{er} and mitochondria become more significant.

The linguistic information entropy of the \gls{tmh} string as well as the GlobProt disorder were also examined.
Significance was found in disorder when comparing \gls{tmh} between \gls{er} and Golgi ($P<5.01E-4$) and the \gls{pm} ($P<2.87E-6$)(Table \ref{table:organellesswissstats}), but there was no observable difference between the \gls{er} and \gls{pm}.
No significance was observed in any consideration of the linguistic entropy, but similarly to the UniProt subset, as a trend the entropy mirrors the hydrophobicity (Figure \ref{fig:average_organelle_factors_ta}).


When considering disorder, we see in both datasets that the Golgi subset is the most negative at -0.2 with \gls{tmh} and -0.25 with the flanks in both the UniProt manually curated and the SwissProt filtered datasets.
Whereas mitochondria is the least negative in both cases when considering the \gls{tmh} (-0.14 in UniProt and -0.13 in SwissProt) or the \gls{tmh} and $\pm$5 flanking residues (-0.18 in UniProt and -0.16 in SwissProt)(Figure \ref{fig:average_organelle_factors_ta}).

Whilst we expect to see hydrophobic adaptations of \gls{tmh} to the membrane environment at both a species and organelle level, we only observe such differences at the organelle level.
% Are the organelle membrane more compositionally different than species?

% \subsection{SNAREs are consistently more hydrophobic across the entire \gls{tmh}}
% Average lines for figure, table for stats

%~\gls{snare} proteins were noted to have more of the most hydrophobic amino acids than the general population of tail anchors~\cite{Kalbfleisch2007}.
%We exploit the difference by mapping the hydrophobicity of the transmembrane domains from a novel list of potential~\gls{ta} proteins generated in this study onto the experimentally validated~\gls{ta} hydrophobicity plot compiled by Kalbfleisch et al. published in Traffic 2007 (8: 1687-1694).
%This method has revealed potential~\gls{ta}~\gls{snare}s with~\gls{snare} motif domains that may not appear in conventional screening.

\subsection{Spontaneous insertion may be achieved by polar patches in the \gls{tmh}}
%Spont TA lines against background dataset.

There are a small group of tail-anchored proteins that have very polar transmembrane domains and are capable of liposome membrane insertion without insertion machinery, also known as spontaneous insertion, that are chaperoned by Hsp70/Hsc40 rather than the TRC40 pathway~\cite{Rabu2008, Rabu2009, Colombo2009}.
Chimeric synaptobrevin, one of the first identified~ \gls{snare} proteins, is capable of spontaneous insertion if the tail anchor domain is replaced by the~\gls{tm} domains belonging to a protein of known spontaneously inserting domains.

\section{Discussion}

We expect to see evolutionary adaptations of \gls{tmh} hydrophobicity to species specific membranes, even within eukaryotes~\cite{Baker2017, Sharpe2010}.
In this study using both a manually curated dataset from UniProt, and an automatically filtered list using SwissProt annotation, we do not observe any strong differences.
Since we could not scrutinise a difference in the species, the strong hydrophobic differences between \gls{tmh}s from different organelles may not be solely accounted for by adaptation to the different membrane compositions.
Given the large biochemical distinction between \gls{ta} proteins with different terminal destinations, it is possible to conclude that \gls{tmh}s contain necessary, yet cryptic, biological factors that play a role in their targeting.
It is indeed possible that our observations are adaptations to the membrane environment, and this would not be unreasonable, except that \gls{ta} proteins would be expected to experience similar adaptations at a species level, for which at this sample size such an effect is unobservable.
This is almost certainly aided by other factors and is part of a system with several redundant mechanisms.

This could indeed be a cryptic functional similarity to the signal anchored proteins.
Signal anchored proteins contain a single hydrophobic segment that serves as both a mitochondrial targeting signal and a membrane anchor.
Interestingly, these proteins, along with some tail-anchored proteins, have been shown to be able to spontaneously insert into the membrane independently from the translocon~\cite{Elisa2012, Lan2000, Colombo2009}.

The idea that \gls{snare} proteins are modular and capable of spontaneous insertion has significant implications for both biomedical application in liposome-based drug delivery and can aid future research for testing complex biological molecular networks~\cite{Allen2013, Nordlund2014}.
