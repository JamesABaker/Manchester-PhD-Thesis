\chapter{Tail-anchored protein discovery} %Mutants outlined + any results from Abbi
\section{Abstract}

\section{Introduction}
This study aims to identify \gls{snare} proteins in eukaryotic proteomes by filtering through large datasets using automatically predicted TrEMBL consensus, and manually annotated SWISS-PROT transmembrane regions. The pipeline generates a list of singlepass proteins with a transmembrane domain close to the C terminal, that are not splice isoforms. A previous study predicted 411 tail anchor proteins \cite{Kalbfleisch2007}.

\section{Methods}

\subsection{Filtering the Uniprot database}
Steps carried out by Kalbfleisch {\it et al.} published in Traffic 2007 (8: 1687\-1694) were recreated using up to date tools. The non\-redundant human dataset of 145,715 proteins from SwissProt and TrEMBL \cite{Kalbfleisch2007, TheUniProtConsortium2014}. 2,478 singlepass proteins were programmatically extracted according to the TRANSMEM count from that list. Then TMDs not within 15AA of the C terminal were removed, resulting in 455 proteins. No splice isoforms were detected according to searching for NON\_TER annotation. 195 proteins of the 411 predicted proteins from the previous study were successfully mapped using the Uniprot mapping tools \cite{Kalbfleisch2007, TheUniProtConsortium2014}. Duplicate IDs from the previously predicted tail anchored protein were removed from the set. The remaining dataset contained XXX proteins.

\subsection{Calculating hydrophobicity}

\section{Results}

\subsection{An up to date tail-anchor dataset}

\subsection{Potential tail-anchored SNARE protein discovery}

\subsection{Biology of spontaneously inserting tail anchored proteins}
