\chapter{Conclusions and outlook}

\sloppy
As the idea of positive residues inside the cytoplasm emerged during the late 1980s, so did the idea of negative residues working in concert with~\gls{tmh} orientation.
It was shown that removing a single lysine residue reversed the topology of a model \textit{E. coli} protein, whereas much higher numbers of negatively charged residues are needed to reverse topology~\cite{Nilsson1990}.
Furthermore, there was no conclusive evidence in the literature for an opposing negatively charged skew~\cite{Granseth2005, Nilsson2005a, Sharpe2010, Baeza-Delgado2013, Pogozheva2013}.
Yet these previous studies in various ways did not account for the rarity of the negatively charged residues relative to the more abundant residues.
In \textit{E. coli} negative residues do experience electrical pulling forces when travelling through the SecYEG translocon indicating that negative charges are biologically relevant~\cite{Ismail2015}, and as mentioned, they did to an extent have the ability to reverse topology \cite{Nilsson1990}.
In this thesis, we have demonstrated that the ``negative\--outside'' skew exists in anchoring~\gls{tmh}s.
Additionally, we have made observations regarding how anchoring \gls{tmh}s are optimised to their membranes.
This new insight into the behaviour of \gls{tmh}s will not only help us understand specific cases in biology, but can also inform synthetic biology regarding \gls{tmp}s.

Post\--translationally inserted \gls{ta} proteins integrate into the membrane independently of the co\--translational ribosome\--translocon complex insertion machinery.
They are also of interest as a liposome drug delivery system due to the role of \gls{ta} \gls{snare} proteins in membrane fusion \cite{Ungar2003, Allen2013, Nordlund2014}.
We found compositional differences between the mitochondrial and secretory localised \gls{tmh}s of \gls{ta} proteins.
Alanine is used more commonly in the mitochondrial \gls{tmh}s, perhaps even more so than isoleucine and leucine.
We also observed a reversal of the positive\--inside and negative\--outside distributions of charged \gls{tmh} flanking residues in mitochondrially localised \gls{ta} proteins.
This re\--emphasises the notion that there is more to \gls{ta} protein trafficking than the hydrophobicity of the \gls{tmh} alone \cite{Guna2018}, and could inform more specific studies of individual proteins.
Still, several important questions remain unanswered that limit our understanding of \gls{ta} protein biosynthesis.
How truly redundant are the different pathways?
What are the precise biophysical and biological signatures of \gls{ta} proteins that determine their path?
Why do some post-translationally inserted \gls{ta} proteins have C-terminal tail lengths exceeding the traditional 25 residue length, whereas other proteins with far shorter tails than this cut-off are co-translationally inserted?
Ultimately, for these questions to be more thoroughly answered, as we have demonstrated, annotation of the protein databases must be completed.

Whilst we have identified amphipathic \gls{tmh}s that may contribute to the spontaneous insertion of some \gls{ta} proteins, \gls{md} simulations of the interactions between the membrane and the anchoring \gls{tmh} of spontaneously inserting \gls{ta} proteins would be needed to reveal the molecular energetic contributions involved in the spontaneous insertion.


The biological scale of hydrophobicity revealed that many \gls{tmh}s do not make the energetic requirements of insertion by the translocon \cite{Hessa2007}.
It is now understood that these so\--called marginally hydrophobic \gls{tmh}s rely on sequence context for correct topology \cite{Ojemalm2012} and efficient insertion \cite{Hedin2010, Junne2017, Virkki2014}.
Here we show that marginally hydrophobic \gls{tmh}s located sequentially adjacent to typically hydrophobic \gls{tmh}s is a conserved feature throughout some families of \gls{tmp} such as ion channels and \gls{gpcr}s.
In these two cases, there is evidence demonstrating that these hydrophobic discrepancy pairs are capable of inserting cooperatively \cite{Ismail2008, Zhang2007, Sato2002, Sato2003, Cymer2015}.
It remains unclear as to what extent the newly discovered role of the \gls{emc} integration of marginally hydrophobic \gls{tmh}s \cite{Shurtleff2018} works in respect to cooperatively inserted \gls{tmh}s.

Understanding cooperative \gls{tmh} insertion and the integration of marginally hydrophobic \gls{tmh}s will require a much more thorough understanding of the temporal and energetic aspects of the process.
It appears that the relatively new technologies of single molecule \gls{em} and \gls{ap}s may be able to grant insight into the energetics and dynamics of the structures throughout the process of \gls{tmp} integration.
Understanding the relationship \gls{tmh}s have with one another is critical for developing complex synthetic \gls{tmp}s.

More generally, there are several challenges still facing our understanding of \gls{tmp}s.
The description of a~\gls{tmh} remains incomplete; it remains difficult to interpret the role of a \gls{tmh} in the membrane beyond anchoring.
Furthermore, the prediction of~\gls{tmp} topology is often erroneous and requires experimental validation.
Despite a wealth of structures the general model of helix\--helix and helix\--lipid interactions remains speculative and requires intensive analysis to generate a working model of a particular~\gls{tmp}.
Nevertheless, progress is being made in our general understanding of \gls{tmp}s.
