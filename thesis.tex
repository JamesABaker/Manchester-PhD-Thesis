% The 12pt option is required by the 2001/02 thesis regulations
% Hacked together from the LaTeX template from the maths department of UoM

%Magic latex symbol finder: http://detexify.kirelabs.org/classify.html

%This is just the standard line that tells the virtual "printer" what document we're making.
% Remove the two side option for single-sided printing
\documentclass[12pt,PhD,twoside]{muthesis}

%These are the citation styles.
\usepackage[backend=bibtex,style=nature,citestyle=nature, doi=false,isbn=false,url=false, maxbibnames=99]{biblatex} %science, nature, alphabetic are common choices.
%FYI

% Citations styles = https://www.sharelatex.com/learn/Biblatex_citation_styles
% Bibliography styles = https://www.sharelatex.com/learn/Biblatex_bibliography_styles

\usepackage{graphicx} %use [!ht] (which means ignore sensible protocols and place here) not [h] (which means place here-ish) for figures otherwise they end up at the end of the chapter.
\usepackage{amsmath}
%\usepackage{siunitx}
\usepackage{csquotes}
\usepackage{longtable}
\usepackage[font=footnotesize,labelfont=bf,justification=justified,format=plain,up,textfont=up]{caption} %\usepackage{tabularx}
\usepackage{ctable}
\usepackage{multirow}
\usepackage{enumerate}
\usepackage[acronym]{glossaries}
\usepackage{ragged2e} % allows handling long code lines and words over multiple lines.
\usepackage{rotating} % allows rotated figures

%Special commands
\newcommand{\specialcell}[2][c]{%
  \begin{tabular}[#1]{@{}c@{}}#2\end{tabular}}
\newcommand{\angstrom}{\textup{\AA}}

\addbibresource{references.bib} % This is the file containing all your bibtex references. All reference managers can export to this format.
\graphicspath{{images/}} %Store all your figures here

% Example abbreviation: {the code used in the text} {the short form} {the long form}. For help on introducing nomenclature, see the sharelatex help page. http://tex.stackexchange.com/questions/86666/how-to-create-both-list-of-abbreviations-and-list-of-nomenclature-using-nomencl

\makeglossaries{}
\newacronym{pm}{PM}{Plasma Membrane}
\newacronym{md}{MD}{Molecular Dynamics}
\newacronym{tm}{TM}{Transmembrane}
\newacronym{tmh}{TMH}{Transmembrane Helix}
\newacronym{tms}{TMS}{Transmembrane Segment}
\newacronym{tmp}{TMP}{Transmembrane Protein}
\newacronym{tmd}{TMD}{Transmembrane Domain}
\newacronym{er}{ER}{Endoplasmic Reticulum}
\newacronym{ta}{TA}{Tail Anchor}
\newacronym{gpi}{GPI}{Glycosylphosphatidylinositol}
\newacronym{popc}{POPC}{Palmitoyloleoylphosphatidylcholine}
\newacronym{pdb}{PDB}{Protein Data Bank}
\newacronym{sp}{SP}{Signal Peptide}
\newacronym{snare}{SNARE}{Soluble N-ethylmaleimide-sensitive factor attachment protein receptor}
\newacronym{ks}{K\--S}{Kolmogorov\--Smirnov}
\newacronym{kw}{K\--W}{Kruskal\--Wallis}
\newacronym{gpcr}{GPCR}{G protein\--coupled receptor}
\newacronym{htl}{HTL}{Holotranslocon}
\newacronym{rna}{RNA}{Ribonucleic Acid}
\newacronym{srp}{SRP}{Signal Recognition Particle}
\newacronym{sr}{SR}{Signal Recognition Particle Receptor}
\newacronym{ap}{AP}{Arrest Peptide}
\newacronym{emc}{EMC}{ER Membrane protein Complex}
\newacronym{em}{EM}{Electron Microscopy}
\newacronym{tom}{TOM}{Translocase of the Outer Membrane}
\newacronym{tim}{TIM}{Translocase of the Inner Membrane}
\newacronym{mom}{MOM}{Mitochondrial Outer Membrane}





\begin{document}
\sloppy

\title{Investigating the Recognition and Interactions of Non-Polar \(\alpha\) Helices in Biology}
\author{James Alexander Baker}
\orcid{orcid.org/0000\--0003\--0874\--2298}
\faculty{Chemistry}
\def\wordcount{22,000}


% Uncomment the line below to suppress the `List of Tables' page (optional)
%\tablespagefalse{}

% Uncomment the line below to suppress the `List of Figures' page (optional)
%\figurespagefalse{}

% Uncomment the line below to use a customised Declaration statement
%\def\declaration{All the work in this thesis has been sourced from Google}


\beforeabstract{} % Don't move the abstract, it doesn't like it. The \beforeabstract command creates the title page, a copyright page (default), the tables of contents, tables and figures.


\prefacesection{Abstract}
Non\---polar helices feature prominently in structural biology.
Transmembrane \(\alpha\) helix containing proteins make up around a quarter of all proteins, represent around 40\% of drug targets, and contain some of the most critical proteins required for life as we know it.
Yet they are fundamentally difficult to study experimentally.
This is in part due to the very features that make them so biologically influential; their non\---polar transmembrane helical regions.

By leveraging large data-sets of transmembrane proteins, this thesis is focused on characterising features of transmembrane \(\alpha\) helices en masse, particularly regarding their topology, membrane\---protein interactions, and intramembrane protein interactions.

In this study, we present statistical evidence demonstrating the ``negative\--outside'' rule in opposition to the ``positive\--inside'' rule.
We also identify stabilising amino acid distributions in anchoring transmembrane helices compared to transmembrane helices with function beyond anchoring.

Tail\--anchored proteins are a group of post\--translationally inserting proteins.
In this thesis we show adaptations of residue distributions through the transmembrane helices of tail\--anchored proteins to different membrane environments within the cell (the mitochondria, endoplasmic reticulum, the Golgi, and the plasma membrane), but could not scrutinise a difference between global populations of tail\--anchored proteins in species (mammals, plants, and fungi).
A handful of these proteins are capable of integrating into the membrane without the need for membrane integration proteins.
Structural modelling of transmembrane helices from PTP1b and cytochrome b5 reveals a 3D amphipathic arrangement of residues.
This structural feature may play a role in their spontaneous membrane insertion.

Finally, we find a conserved pattern of typically hydrophobic transmembrane helices neighbouring marginally hydrophobic helices in some families of transmembrane proteins.
This feature corresponds to transmembrane helices that have the potential to cooperate in order to integrate the more polar, but functionally important, transmembrane helix of the pair into the membrane.


\prefacesection{Lay Abstract} %The lay abstract section is optional.

The survival of each of our cells relies on a cellular barrier (called the membrane) to separate themselves from the surrounding environment.
The membrane works by being chemically very different from both the outside environment and the inside of the cell, which in both cases is mostly water.
The membrane is fatty so repels water.

Proteins are the molecular machinery that forms much of the cell structure and shape as well as carrying out many of the cell's routine tasks.
Around a third of our genome codes for membrane\--embedded proteins.
But because these membrane\--proteins are adapted for a life in the water\--repelling cell membrane, they are very hard to study in laboratories which often rely on methods that hold proteins in water\--based solutions.

In this thesis, we focus particularly on the parts of the protein that are embedded in the water repelling membrane.
We computationally analysed the biochemical make\--up of thousands of proteins from openly available biological databases.

This thesis demonstrates three features of membrane proteins:
\begin{itemize}

  \item the radically different evolutionary story that membrane\--bound regions have compared to other proteins; the sacrifices they make for their stability in order to maintain their function, and their optimisation through evolutionary timescales to adapt to the membrane as best they can.

  \item a distinct sub\--group of membrane\--proteins that have a radically different membrane\--insertion mechanism (tail\--anchored proteins) have adaptations in their membrane regions depending on where they are located in the cell.

  \item some types of membrane proteins may use several membrane elements to ensure the least stable, but functionally important, elements are correctly inserted into the membrane.

\end{itemize}

These results will go on to inform more specific studies about membrane proteins.
These findings will provide insight into the causes of some genetic diseases as well as drug targets in the case of pathogenic infections and cancers.



 %But what is the point, what have we learned.



\afterabstract{} % \afterabstract command to insert the declaration.

%The preface section doesn't like being in a different document for some reason. (I think it's just my compiler, there is no reason why it shouldn't work)
\prefacesection{Acknowledgements}
I would like to thank all members of both the Eisenhaber research group, as well as the Curtis and Warwicker research group for their interesting and engaging discussion throughout my PhD.
In particular I would like to acknowledge the patience, guidance, and exceptional supervision from Dr Jim Warwicker, and Professor Stephen High of the University of Manchester and Dr Frank Eisenhaber, and Dr Birgit Eisenhaber, from the A*STAR in Singapore.
I also express my gratitude towards Dr Wing Cheong Wong and Dr Max Hebditch for working with me on many fiddly problems.
I would also like to thank The University of Manchester and the ARAP programme at the Singapore Bioinformatics Institute for funding the project.

Outside of the office I thank my mother, father, my brother Tim, and Emily for their unwavering support of me in undertaking a PhD based for two years on the otherside of the planet.
Being so far from my family and loved ones was certainly the most challenging part of this process.

%It was my uncle Howard's wisdom and philosophical outlook that swayed me to undertake a PhD.
%May he rest in peace.

% \prefacesection{The Author}

% In 2013, I undertook an internship at Chulalongkorn University under the supervision of Dr. Thanyada Rungrotmongkol. The internship contributed to simulating several compounds that could target a viral protease (\(3C^{pro}\)) of both Cocksackie 16 and Enterovirus 72, of which infect X and kill Y people per year. The results showed that several compounds were more potent inhibitors at a protein level than the currently championed drug candidates.

% I have previously completed a Masters in Biochemistry from the University of Liverpool, graduating in 2014. In the final year of the Masters degree, under the supervision of Dr. Dan Rigden, I modelled domains of a transmembrane protein involved in bacterial competence (ComEC) in \textit{Bacillus subtilis}. Structural and sequential analysis of the models revealed a cryptic OB fold in a domain of unknown function, and a potential competence nuclease function in the $\beta$--lactamase--like domain~\cite{Baker2016}.

% During the first year of this PhD studentship, the author presented a part of the work in Berlin at the FEBS annual congress (Baker, J. A. and Warwicker, J. A Bioinformatic Method to Identify Potential SNARE Proteins. \textit{ 40th FEBS Congress} Late Breaker (2015))

%%%%%%%%%%%%%%%%%%%%%%%%%%%%%%%%%%%%%%%%%%%%%%%%%%%%%%%%%
%\subsection*{Preamble} %A side note. the "*" hides the section number.
%In the 1950s and 1960s, the field of biological philosophy was still emerging. David Hull writes on the matter in his 1969 article entitled \textit{``What Philosophy of Biology is not''}:

%\begin{displayquote}
%``Periodically through the history of biology, biologists have tried to do a little philosophy...''\cite{Hull1969}
%\end{displayquote}

%I can't help but wish my thesis title was ``The ins-and-outs of greasy peptides''.
%%%%%%%%%%%%%%%%%%%%%%%%%%%%%%%%%%%%%%%%%%%%%%%%%%%%%%%%%

%%% List of publications not necessary.
% \prefacesection{List of publications}

% \subsection*{Journal Articles} % the * hides the section number

% \subsection*{Posters}
%Baker, J. A. and Warwicker, J. A Bioinformatic Method to Identify Potential SNARE Proteins. \textit{ 40th FEBS Congress} Late Breaker (2015)

%\clearpage
\afterpreface{}

%\pagenumbering{arabic}

%This is where we call the chapters. No need to include ".tex". The chapters will be printed to the document in this order. To create a new chapter, make a new file, and add the name of the file below (without the file extension).
\chapter{Introduction}


\section{The Transmembrane Protein Problem}
%Kyte and doolittle

The insertion and formation of the unusually orientated \gls{tmh}s and of the more traditional \gls{tmh}s have been shown to be underpinned by complex thermodynamic equilibria \cite{Cymer2014}. \gls{tmh}s have been identified as regulators of protein quality control and trafficking mechanisms, shifting the idea away from \gls{tmh}s broadly simply functioning as anchors \cite{Hessa2011}. The story is not as simple as originally thought. There is a contingency in the field of biological membranes that despite progress over the last decade, there is a still lack of information regarding their structure, assembly, and the behaviour of \gls{tmh}s in the lipid bilayer; the native biological environment of \gls{tmh}s \cite{Ladokhin2015}.


\section{Biological Membrane Composition}

%Something about aysmetry, varies through the tree of life. "before we discuss the membrane proteins, one must consider the biological reason as for why they exist."
%to do:  give specific examples of incredibly useful membrane protiens. Outline that MPs are vital for relaying information and chemistry across the membrane.


%At some point the membrane phases should be discussed. Similar to the diagram here perhaps: http://popups.ulg.ac.be/1780-4507/index.php?id=6568
The compartmentalisation of cellular biochemistry is arguably one of the most significant events to have occurred in evolution, and is certainly one of the fundamental prerequisites for life \cite{Koshland2002}.  The proteins that allow life to use this biochemical barrier are perhaps equally important. Together, the lipid bilayer and proteins therein allow complex biochemical systems that facillitate life to exist.

It is critical to understand that the lipid bilayer and the transmembrane $\alpha$ helices are inextricably linked, and often what we observe from the $\alpha$ helices reflect the properties of the much harder to study membranes. The lipid membranes influence the local structure, dynamics, and activity of proteins in the membrane in non-trivial ways \cite{Bondar2010, Bondar2009, Jardon-Valadez2010, Kalvodova2005, Urban2005, White2001, Jensen2004, Henin2014}.%Henin2014. %Add references about the hydrogen bonds (white2005 and another one...) %Perhaps this wedge of citations should be expanded.

It has been known for some time that the outer membranes of Gram negative bacteria are asymentric in terms of lipid composition. The outer membranes contain lipopolysccharide, whilst the inner is a mixture of approximately 25 phospholipid types. Adding to the membrane asymmetry composition story, a thorough analysis of residue composition in yeast and human \gls{tmh} regions revealed intra-membrane leaflet composition asymetry in the \gls{er}, but not the Golgi \cite{Sharpe2010}. Furthermore protein\-lipid interactions have been shown to be determinants of membrane curvature \cite{Jensen2004}, and undertake complex orientations and conformations to allow for hydrophobic mismatch \cite{Planque2003}. %the planque reference is messed up. It may need changing with every bib.tex update unless the permannet record is changed.

There is a rich variety of lipid molecules that make up the biological membranes. The majority of lipids in higher organism membranes are phospholipids, sphingolipids, and sterols. These are composed of a glycerol molecule. Bonded to the glycerol molecule are two hydrophobic fatty acid tail groups, and a negatively-charged polar phosphate group. The polar phosphate group is modeified with an alcohol group. Water entropicly drives the self association of the lipid molecules. In other words the bilayer forms from these phospholipid molecules due to the fierce dissociation between the polar water and the hydrophobic tails. Furthermore the bilayer maximises van der Waals interactions between the closely-packed hydrocarbon chains, which contributes to the stability of the bilayer. This can be seen even in relatively early \gls{md} simulations \cite{Goetz1998}.

\section{$\alpha$ Helices in Membranes }
\subsection{The Importance of Transmembrane Proteins}
Membrane bound proteins underpin almost every biological process directly, or indirectly, from photosynthesis to respiration. Integral \gls{tmp} are encoded by around 30\% of the genes in the human genome which reflects their biological importance \cite{Almen2009}. These proteins allow biochemical pathways that traverse the various biological membranes used in life. %citation for drug targets

%This should basically be a summary of Pogozheva, Baeza-delgado, and  Sharpe. Cite examples of direct and indirect accurate prediction methods

%Wong papers on how the sequences contain nuanced signals.

%Function is a result of structure, and here the structure and sequence are supposedly very relatable.
The relationship between the membrane and \gls{tmp}s is underpinned by complex thermondynamic and electrostatic equilibria. Once inserted the protein doesn't leave the membrane as a result of the transmembrane helix being very hydrophobic. This hydrophobicity, and the hydrophobicity of the lipid tails means that they self associate. A better way of describing it is that they fiercely dissociate from the water. The overall $\Delta$G for a transmembrane helix in the membrane is -12kcal${mol}^{\--1}$ \cite{Cymer2014}; the association of the helix in the membrane is typically spontaneous.

\subsection{Transmembrane Helix Sequence Composition}

%WALP and KALP peptides as typical peptides
\begin{figure}[!ht]
\centering
\includegraphics[width=1\textwidth]{Helix_anatomy}
\caption{A cartoon showing the general components of the membrane and a typical \gls{tmh}. The example used here for illustrative purposes is transmembrane region of tetherin (PDB 2LK9) \cite{Skasko2012}. Dark grey areas denote the area of lipid head groups. The residues found in these areas are often described as flanking regions, and are often in contact with the aqueous interface of the membrane. The helix core is mostly composed of hydrophobic residues. More recently the hydrophobic group region has been associated with cell localisation and a broad range of biochemical functions \cite{Junne2010, Wong2012}. Although the regions labelled here generally hold true in terms of the statistical distribution of polar, non-polar, and charged groups, it is by no means absolute laws and many proteins break these ``rules'' \cite{Sharpe2010, Baeza-Delgado2013, Pogozheva2013}. Note that the definition of a \gls{tm} $\alpha$-helix is not entirely clear; how far the helix rises into the water-interface region to qualify as a \gls{tmh} for example \cite{VonHeijne2006}.}
\label{fig:helixcartoon1}
\end{figure}


% Figure: A cartoon depicting various problematic, yet biologically observed topologies and lengths that the alpha helices can adopt. From left to right: a typical and traditional \gls{tmh}, an exceptionally long \gls{tmh}, a \gls{tmh} that lies flat in the interface region, a kinked helix that enters and exits the bilayer on the same leaflet, a \gls{tmh} that is not long enough to span the entire membrane. Note that these exceptional formations present a challenge for topology predictions of the loop regions.}


Properties that can be analysed by bioinformatics, the sequence complexity and hydrophobicity, of the \gls{tmh} have been used to predict the role of the \gls{tmh} as either functional or structural, and as a discrete cluster from other SCOP annotated helices \cite{Wong2012}. Those findings demonstrated that the sequence of the \gls{tmh} holds valuable information regarding biological roles, and forms the basis of our interest in the link between the polarity of a helix and functional activity beyond structural anchorage.

The language used to describe \gls{tmh}s varies somewhat across the literature, primarily due to a changing understanding of \gls{tmh} general structure and relevance to function over the last 15 years or so. There is a general composition of a \gls{tmh} despite specific protein and membrane constraints \cite{Sharpe2010}.

%This paragraph should certainly be changed to the updated one from the manuscript.
A study by Baeza-Delgado {\it et al.} from 2013 \cite{Baeza-Delgado2013} looked at \gls{tmh}s in 170 integral membrane proteins from a manually maintained database of experimentally confirmed \gls{tmp}s; MPTopo \cite{Jayasinghe2001}. The group examined the distribution of residues along the \gls{tmh}s. As expected, half of the natural amino acids are equally distributed along Transmembrane (TM) helices whereas aromatic, polar, and charged amino acids along with proline are biasedly near the flanks of the TM helices \cite{Baeza-Delgado2013}. It has been noted that transitions between the polar and non-polar groups at the ends of the hydrophobic core occur in a more defined edge on the cytoplasmic side than at the extracytoplasmic face when counting from the middle of the helix outwards \cite{Baeza-Delgado2013}. This is probably reflecting the different lipid composition of both leaflets of biological membranes \cite{Baeza-Delgado2013}. A larger previous study using 1192 human and 1119 yeast predicted \gls{tmh}s that were not structurally validated further explored the difference in \gls{tmh} and leaflet structure by exploiting the evolutionarily conserved sequence differences between the \gls{tmh} in the inner and outer leaflets \cite{Sharpe2010}. \gls{tmh}s from vertebrates and invertebrates were found to be reasonably similar compositionally. The differences in consensus \gls{tmh} structure implies that there are general differences between the membranes of the golgi and \gls{er}. The abundance of serines in the region following the lumenal end of golgi TMDs probably reflects the fact that this part of many golgi enzymes forms a flexible linker that tethers the catalytic domain to the membrane \cite{Sharpe2010}.

\subsubsection{The ``Positive-Inside'' Rule}

Two publications by von Heijne coined the ``Positive-Inside'' rule demonstrated the practical value of positively charged residue sequence clustering in topology prediction of \gls{tmh}s in bacteria \cite{VonHeijne1989,VonHeijne1992}. It was clearly defined and shown that positively charged residues more commonly were found on the ``inside'' of the cytoplasm rather than the periplasm of {\it E. coli}. More recently still large scale sequence analysis of \gls{tmh}s from different organelle membrane surfaces in eukaryotic proteomes, show the clustering of positive charge being cytosolic \cite{Sharpe2010, Baeza-Delgado2013, Pogozheva2013}.

\subsubsection{The Aromatic Belt}

 Tyrosine and tryptophan residues commonly are found at the interfacial boundaries of the \gls{tmh} and this feature is called the ``aromatic belt'' \cite{Hessa2005, Granseth2005, Sharpe2010, Baeza-Delgado2013, Nilsson2005}. Not all aromatic residues are not found in the aromatic belt; phenylalanine has no particular preference for this region \cite{Granseth2005, Braun1999}. However it still remains unclear if this is to do with anchorage or translocon recognition \cite{Baeza-Delgado2013}.

%More than one explanation has been proposed for the physical and/or chemical basis for the strong preference of aromatic residues to reside in the membrane-water interface region. The Tyrosine side chain is a six-membered aromatic ring with an –OH group attached. Tryptophan has two aromatic rings that are fused into one large hydrophobic ring-structure. Phenylalanine, although aromatic, is completely hydrophobic, and is found in the transmembrane part rather than the interfacial parts of MPs. The classical explanation for the preference of Tyrosine and Tryptophan to reside in the interfacial regions is their dipolar character. The side chain must simply seek a compromise. This can be achieved by burying the aromatic ring close to, or within, the hydrophobic core, while the hydrophilic part can interact with the polar lipid head-groups at the interface. Other factors such as the aromaticity, size, rigidity and shape of Tryptophan, rather than its dipolar character, has also been suggested as the primary reasons for its interfacial preference [91]. Perhaps, as suggested by You et al., it is a balance of all these forces that explains the interfacial preference of Tyrosine and Tryptophan [91

\subsubsection{Snorkeling}

Broadly speaking, transmembrane helices are non-polar. However some contain polar and  charged residues in the helix itself. Whilst this might seem thermodynamically unstable at first glance, a molecular dynamic feature called the ``snorkel'' effect explains in part how this is possible \cite{Chamberlain2004, Strandberg2003}. Simply put, the snorkelling effect involves the long flexible side chain of leucine reaching the water interface region to interact with the polar headgroups of the bilayer even when the $\alpha$ helix backbone is pulled into the hydrophobic layer \cite{Krishnakumar2007}. This has also been suggested to allow helices to adapt to varying thicknesses of the membrane \cite{Kandasamy2006}.

\section{Biogenesis of Transmembrane Proteins}
%Depth of secretion pathway (talk about translocons in depth here) & outline post-translational insertion
\subsection{Translocation}
The ``inside'' was an imprecise term used to indirectly refer to the cytoplasmic space. To understand why the cytoplasm is the key part, one must recall how the membranes are synthesised and localised throughout the cell.

\subsection{Tail-Anchored Proteins Post Translationally Insert}

Tail anchored proteins are a topologically distinct class of intracellular proteins defined by their single carboxy-terminal transmembrane domain with a cytosolic facing amino-terminus. Tail anchored proteins are involved in a range of key cellular functions including protein translocation and apoptosis. Additionally, within the tail anchored class of proteins are a set of vesicle fusion proteins called \gls{snare} proteins. There is biomedical interest in \gls{snare} drug delivery mechanisms. \gls{snare}s can fuse liposomes containing various drug payloads into the membrane.

Notably, known \gls{snare} transmembrane helices are highly hydrophobic even compared to other tail anchored transmembrane helices.

\subsection{Translocon Independent Membrane Insertion}
%move on to case studies involving spontaneous insertion?
Signal anchored proteins, proteins that contain a single hydrophobic segment that serves as both a mitochondrial targeting signal and a membrane anchor, as well as tail anchored proteins have been shown to be able to spontaneously insert into the membrane indepently from the translocon \cite{Elisa2012, Lan2000, Colombo2009}.

It is postulated that there are electrostatic factors in the flanking regions that contribute to this spontaneous membrane insertion. Our experimental collaborators in Stephen High’s group are interested in a small group of tail anchored proteins that have very polar transmembrane domains and are capable of liposome membrane insertion without insertion machinery, also known as spontaneous insertion. They have found that chimeric synaptobrevin, one of the first identified \gls{snare} proteins, is capable of spontaneous insertion if it’s tail anchor domain is replaced by the transmembrane domains belonging to a protein of known spontaneously inserting domains. Their studies have moved the focus of spontaneous insertion away from the loop regions and onto the biophysicochemical factors of the TMH itself. The idea that \gls{snare} proteins are modular, and capable of spontaneous insertion has significant implications for both biomedical application in liposome based drug delivery and can aid future research for testing complex biological molecular networks \cite{Allen2013, Nordlund2014}.

\section{Choice of Hydrophobicity Values}
\subsection{An Overview of the Different Scales}
Throughout this thesis several scales are used to evaluate and estimate hydrophobic values of peptides. All the scales aim for quantifying the hydrophobic values of each residue. There are several key differences in their methodology, assumptions, and aims. Crucially this results in slightly different scores for some residues. Because of this, it's preferable, and typical amongst the literature, to use several scales to verify the patterns observable in one scale. Notably, one of the classic scale, Kyte \& Doolittle Hydropathy Scale shows striking similarity to the modern Hessa's Biological Hydrophobicity Scale, and that generally the ``better'' scales count proline as hydrophilic, and focus on helix recognition rather than amino acid analogues \cite{Peters2014}. Ultimately, all the scales are attempting to allow estimation of ${\Delta G}_{whf}$; the free energy of a folded helix ({\it f}) from the water ({\it w}) into the membrane core ({\it h}). This free energy measurement is regarded as being currently experimentally inaccessible \cite{Cymer2014}.

%Figure here showing what delta G is on about
%Also a note on arrest peptides

\subsubsection{Kyte \& Doolittle Hydropathy Scale}
A scale based on the water\---vapour transfer free energy and the interior-exterior distribution of individual amino acids \cite{Kyte1982}.

\subsubsection{Hessa's Biological Hydrophobicity Scale}
This is arguably the most biologically relevent scale \cite{Peters2014}. The scale is based on an experimental method where the free energy exchange during recognition of designed polypeptide helices by the endoplasmic reticulum Sec61 translocon occured \cite{Hessa2005}. These measurements were then used to calculate a “biological hydrophobicity scale.”

\subsubsection{White and Wimley Octanol \--- Interface Whole Residue Scale}
This scale is calculated from two other scales; the octanol scale, and the interface scale \cite{White1999}. This scale is fundamentally based on the partitioning of host-guest pentapeptides (acetyl-WL-X-LL-OH) and another set of peptides (AcWLm) between water and octanol, as well as water to \gls{popc} .


\subsubsection{The Eisenberg Hydrophobic Moment Consensus Scale}
The Eisenberg scale is a consensus scale based on the earlier scales from Tanford \cite{Nozaki1971}, Wolfenden \cite{Rose1993}, Chothia \cite{Chothia1976}, Janin \cite{Janin1979}, and the von Heijne scale \cite{VonHeijne1979}. The scales are normalized according to serine \cite{Eisenberg1984}. The automatic TRANSMEM annotation currently used in Uniprot is according to TMHMM \cite{Krogh2001}, Memsat \cite{Jones2007}, Phobius \cite{Kall2004} and the hydrophobic moment plot method of Eisenberg and coworkers \cite{Eisenberg1984}.


\section{A Brief History of Transmembrane Proteins in Science}
\subsection{Earliest Evidences of Compartmentalisation}
%To get started https://en.wikipedia.org/wiki/History_of_cell_membrane_theory
%A group linking computational biophycs http://www.ks.uiuc.edu/History/membrane/

%Some waffle about Hooke

\subsection{Early Models of the Bilayer}
%Gorter & Grendel
%Danielli and Davson http://biology.stackexchange.com/questions/20264/why-was-the-davson-danielli-model-rejected/24051#24051

\subsection{The Rise of Crystallography}
%first crystals in the membrane

Because of the experimental hinderence, the story of transmembrane proteins has been relatively slow to emerge. In the 1990s and early 2000s the story was seemingly uncomplicated. There were membrane-spanning bundles of non-polar α-helices of roughly 20 residues length, with a consistent orientation of being perpendicular to the membrane surface. Since the mid-2000s the elucidation of many more intramembrane helix structures implied a far richer variety of transmembrane helices existed than previously thought, with a range of orientations and intra-membrane biophysical variations. Although the simple helices are broadly prevelant, hundreds of high quality membrane structures have elucidated that \gls{tmh}s can adopt a plethora of lengths and orientations within the membrane. \gls{tmh}s are capable of partial spanning of the membrane, spanning using oblique angles, and even lying flat on the membrane surface \cite{VonHeijne2006, Elofsson2007} (Figure \ref{fig:helixcartoon1}). Over the last decade, Nanodiscs have been routinely used to much more easily obtain crystal structures. Nanodiscs overcome some of the major challenges caused by the hydrophobic helices, and a more faithful representation of the biological membranes than alternative model membranes like liposomes \cite{Borch2009}. %Figure for Nanodisc
%Development of Nanodiscs

%At the time of writing, how many in PDB.

\section{Aims of This Thesis.}

\begin{enumerate}
  \item Negative not inside rule
  \item \gls{gpi} project
  \item \gls{snare} and \gls{ta} project
  \item Good and bad helices
\end{enumerate}


\chapter{The ``Negative-Outside'' Rule}

The description of a \gls{tmh} remains incomplete. The understanding of \gls{tmp} topology is erroneous, and despite a wealth of structures, the general model of helix-helix and helix-lipid interactions remains speculative and requires a great deal of intensive analysis to generate a working model of a particular \gls{tmp}.

The work presented in this chapter is an expanded version of published work\cite{Baker2016}. We use advanced statistical analysis to analyze large sequence datasets that have rich topological annotation. By analyzing these sequences in the context of anchorage, we find that some \gls{tmh}s are confined to biological constraints of the membrane, whereas others that likely contain function beyond anchorage, are less conforming to the membrane. Specifically, there is further elaboration of statistical definitions in the methods than in the published paper. 



\section{Abstract}

\section{Summary}
As the idea of positive residues inside the cytoplasm emerged during the late 1980s, so did the idea of negative residues working in concert with \gls{tmh} orientation. It was shown that removing a single lysine residue reversed the topology of a model {\it Escherichia coli} protein, whereas much higher numbers of negatively charged residues are needed to reverse topology\cite{Nilsson1990}. One would also expect to see a skew in negatively charged distribution if a cooperation between oppositely charged residues orientated a \gls{tmh}, however there is no conclusive evidence in the literature for an opposing negatively charged skew\cite{Granseth2005, Nilsson2005, Sharpe2010, Baeza-Delgado2013, Pogozheva2013}. However, in {\it E. coli} negative residues do experience electrical pulling forces when traveling through the SecYEG translocon indicating that negative charges are biologically relevant\cite{Ismail2015}.

\section{Methods}

\subsection{Normalisation}

$c_r=\frac{(a_{K,r}+a_{R,r})-(a_{D,r}+a_{E,r})}{N}$

$p_{i,r}=\frac{a_{i,r}}{\underset{r}{\max}{(a_r)}}$

$q_{i,r}=\frac{100·a_{i,r}}{a_i}$

\section{Results}
\subsection{Biophysicochemical differences in multi-pass and single-pass helices}

ope\chapter{Tail-Anchored Protein Datasets}
\sloppy
% gls refers to a glossary term, cite refers to an entry from the separate bibtex folder. url is a lazy way of forcing monospaced text. textit is italics. sections and subsections are the headers.
\section{Abstract}

\section{Introduction}

~\gls{ta} proteins are a topologically distinct class of intracellular proteins defined by their single carboxy-terminal~\gls{tms} with a cytosolic facing amino-terminus.
~\gls{ta} proteins are involved in a range of key cellular functions including protein translocation and apoptosis.
Additionally, within the~\gls{ta} class of proteins are a set of vesicle fusion proteins called~\gls{snare} proteins.
There is biomedical interest in~\gls{snare} drug delivery mechanisms.
~\gls{snare}s can fuse liposomes containing various drug payloads into the membrane.

The pipeline generates a list of singlepass proteins with a transmembrane domain close to the C terminal, that are not splice isoforms.
A previous study by Kalbfleisch \textit{et al.} published in Traffic 2007 (8: 1687-1694) predicted 411 tail anchor proteins~\cite{Kalbfleisch2007}.
The tools developed herein are openly available for re-application to other datasets.
Notably, known~\gls{snare} transmembrane helices are highly hydrophobic even compared to other~\gls{ta} transmembrane helices.
We compare Kyte and Doolittle hydrophobicity profiles of our filtered human protein list against the profiles of previously known~\gls{snare} and~\gls{ta} proteins.
This provided a list of potential~\gls{snare} proteins in addition to potential spontaneously inserting~\gls{ta} proteins similar to cytochrome b5 which have the least hydrophobic transmembrane helices.

Tail-anchored proteins are a topologically distinct class of intracellular proteins defined by their single carboxy-terminal transmembrane domain with a cytosolic-facing amino-terminus.
%Need a figure here of a crystal structure TA.

Tail-anchored proteins are involved in a range of key cellular functions including protein translocation and apoptosis.
Additionally, within the tail-anchored class of proteins are a set of vesicle fusion proteins called \gls{snare} proteins.
There is biomedical interest in \gls{snare} drug delivery mechanisms.

\gls{snare}s can fuse liposomes containing various drug payloads into the membrane.
Notably, known \gls{snare} \gls{tmh}s are highly hydrophobic even compared to other tail anchored \gls{tmh}s~\cite{Kalbfleisch2007}.
This hydrophobicity appears to be a determinate factor in the precise delivery mechanistic route that a~\gls{ta} proteins use for insertion~\cite{Rabu2008, Rabu2009}, for which there is evidence demonstrating that are several mechanisms~\cite{Rabu2009, Johnson2013}.

Whilst most eukaryotic~\gls{ta} proteins are inserted into the~\gls{er}.

\section{Results}

\subsection{Tail-Anchored Protein Datasets Are A Moving Target}
Here, we use two sources for \gls{ta} protein datasets.
One dataset is based on a previous method~\cite{Kalbfleisch2007} to obtain \gls{ta} datasets and consists of 9296 \gls{tmh} residues (13279 including up to $\pm$5 flanking residues) from 443 SwissProt entries with 90\% redundancy removal.
Another dataset contains the UniProt curated set of Type IV membrane proteins again with 90\% redundancy removal.
This dataset contains 21119 \gls{tmh} residues (28791 including uptt to $\pm$5 flanking residues) from 987 UniProt protein records.

We compared redundant versions of these two datasets to the S1 dataset from a previous method~\cite{Kalbfleisch2007}, that aimed to gather \gls{ta} proteins in the human genome from the NCBI, to see how many records are shared, how many are now obselete, and how many are unique.

% Venn diagram
\begin{figure}[!ht]
\centering
\includegraphics[width=0.5\textwidth]{TA_chapter/database-overlap}
		\captionof{figure}[A venn diagram showing tail anchored protein UniProt ids present in each of the datasets as well as those present in multiple datasets.]{\textbf{A venn diagram showing tail anchored protein UniProt ids present in each of the datasets as well as those present in multiple datasets.}
The number of ids present in redundant versions of
i) the supplementary materials table of a previous study predicting the complete set of human tail anchored proteins denote by S1~\cite{Kalbfleisch2007},
ii) the Swissprot dataset filtered according to typical~\gls{ta} features limited to the human proteome~\cite{TheUniProtConsortium2014}, and
iii) The UniProt curated list of~\gls{ta} proteins~\cite{TheUniProtConsortium2014}.
Note that to avoid losing IDs to redundancy reduction this diagram was generated without the use of CD-HIT~\cite{Huang2010, Wu2011}, which is applied in later statistical analysis.}

\label{fig:tadatasetoverlap}
\end{figure}

Figure~\ref{fig:tadatasetoverlap} shows that already a study from 2007~\cite{Kalbfleisch2007} has 175 record ids of 222 records (78.8\%) that do not share overlap the up-to-date manually curated UniProt dataset~\cite{TheUniProtConsortium2014}.
Of the 166 unique records of that 2007 dataset, 92 records do have location annotation in UniProt that the scripts herein use for topological determination, leaving 74 records without location annotation.
This leaves 92 of 222 (41.4\%) records that originally fitted criteria that no longer fit those same criteria.
If we exclude those lacking suitable annotation i.e ids from S1 that are found in either SwissProt with the filters (9), the curated UniProt list (14), or both (33), compared to the 92 that have annotation contradicting the original prodictions, 37.8\% of the ids overlap.

Equivalent criteria were applied to the entire SwissProt database and then restricted to the human proteome dataset.
43 of these 77 records (55.8\%) are in the curated UniProt~\gls{ta}  dataset leaving 34 records that meet the criteria out of the manually curated set (44.2\%of the filtered Swissprot dataset).
42 of the 77 (54.5\%) records from SwissProt filtered human dataset can be found in the original S1 list.
A further consideration is that, after removing redundant proteins, this method picked up  46 Archaeal and 66 bacterial records.

Datasets are a moving target as they are constantly updated with more accurate and reliable tools.
Perhaps unsurprisingly, as a trend, this shows that up-to-date datasets improve the reliability of this automated predicted method and that there is a large degree of what we now believe to be mistakes that occured in older prediction tools.
These automated criteria still do not fully align with the manually curated list, which is bound to change too, especially considering only 973 of a non-redundant (90\% CD-HIT threshold \cite{Huang2010, Wu2011}) set of those 2460 proteins of the UniProt manually curated set contained annotation for the transmembrane boundary residues.

Note that these numbers are not absolutely certain.
The greatest source of uncertainty here is that the original S1 list includes 411 records, however only 222 of these were successfully mapped to the UniProt dataset.
This figure is closer to the 202 proteins from the original list that excluded proteins that were either hypothetical or splice isoforms.
That being said, this ``conversion'' step prevents us from directly comparing the entire original S1 dataset.

\subsection{Species Variation}
% Average lines for figure, table for stats
\begin{figure}[!ht]
\centering
\includegraphics[width=1\textwidth]{TA_chapter/species-hydrophobicity}
\captionof{figure}[Average values of species datasets from UniProt manually curated set and SwissProt automatically filtered dataset.]
{\textbf{Average values of species datasets from UniProt manually curated set and SwissProt automatically filtered dataset.}

The average hydrophoicity values from the Kyte \& Doolittle scale ~\cite{Kyte1982} and the GlobProt scale \cite{Linding2003} for both the \gls{tmh} and the \gls{tmh}$\pm$5 residues. Values are shown for both the UniProt manually curated set and the SwissProt filtered set. In the UniProt manually curated set we compare the mammalian set of \gls{ta} proteins (Human N=38 and Mouse N=37) to \textit{A. thaliana} (N=60) representing plants  and  \textit{S. cerevisiae} (N=31) representing yeasts. For the SwissProt filtered set we compare the mammalian set of \gls{ta} proteins (Human N=46 and Mouse N=48) to \textit{A. thaliana} (N=49) representing plants  and  \textit{S. cerevisiae} (N=24) representing yeasts.
Error bars are shown at $\pm 1 \sigma$ from the mean of the respective dataset.
}

\label{fig:average_species_hydrophobicity_ta}
\end{figure}

When comparing the average Kyte \& Doolittle~\cite{Kyte1982} hydrophobicity values for the~\gls{tmh}s from humans and mice, \textit{A. thaliana}, and  \textit{S. cerevisiae}, we can see little difference between the mean values.
All of the mean values lie between $\sim$2.3-2.6 when we only consider the \gls{tmh} and at $\sim$1.3-1.6 when considering residues in close proximity to the~\gls{tmh} ($\pm$5 residues) (Figure~\ref{fig:average_species_hydrophobicity_ta}).
Indeed, we see no strong observable statistical differences in hydrophobicity ($P-value>8.30E-02$ in the UniProt curated list Table \ref{table:speciestableuniprotstats}, $P>3.35E-01$ in the SwissProt automatically filtered list Table \ref{table:speciestableswissprotstats}).

In single-pass proteins of Eukaryotic species there are typically various adaptations of the \gls{tmh} to adhere to the membrane constraints of the specific membrane that can be observed in terms of hydrophobicity~\cite{Sharpe2010}, especially in \gls{tmh} anchors~\cite{Baker2017}.
However, in these \gls{ta} protein datasets we see no such patterns at this sample size.

Here, we are dealing with datasets at least an order of magnitude smaller than those broad studies which could explain the absence of the effect.
However this only goes to show that if there is an effect in \gls{ta} proteins, it is indeed weak.

\begin{table}[htbp]
\centering
\captionof{table}[Hydrophobicity statistical comparisons between mouse and human, yeast, and plants in the SwissProt Filtered Dataset.]
{\textbf{Hydrophobicity statistical comparisons between mouse and human, yeast, and plants in the SwissProt Filtered Dataset.}
Here, we compare a mammalian set of \gls{ta} proteins (Human N=46 and Mouse N=48) to \textit{A. thaliana} (N=49) representing plants  and  \textit{S. cerevisiae} (N=24) representing yeasts.
The hydrophobicity was predicted as the mean average of the values of the sequences of the \gls{tmh}, as well another group including up to $\pm$5 flanking residues predicting the boundary of \gls{tmh}s is difficult, according to the Kyte \& Doolittle hydrophobicity scale~\cite{Kyte1982}.
%Disorder was calculated in the same way using the GlobProt scale \cite{Linding2003}.
The Test column refers to the statistical score obtaineed from the test; H statistic for the Kruskal Wallis, the KS statistic for the Kolmogorov Smirnov test, and the t-statistic for the T-test.
$P$ is the P-value of that statistical score.
$B$ refers to the Bahadur slope, an interpretation of the P-value that accounts for the sample size powering the test~\cite{Bahadur1967, Bahadur1971}.}
\tiny
	% Table generated by Excel2LaTeX from sheet 'SwissProt filtered species'

    \begin{tabular}{clrrrrrrrrr}
          &       & \multicolumn{3}{c}{Mammal and Plant} & \multicolumn{3}{c}{Mammal and Yeast} & \multicolumn{3}{c}{Plant and Yeast} \\
          &       & \multicolumn{1}{l}{ Test} & \multicolumn{1}{l}{ $P$} & \multicolumn{1}{l}{ $B$} & \multicolumn{1}{l}{ Test} & \multicolumn{1}{l}{ $P$} & \multicolumn{1}{l}{ $B$} & \multicolumn{1}{l}{ Test} & \multicolumn{1}{l}{ $P$} & \multicolumn{1}{l}{ $B$} \\
    \multirow{3}[0]{*}{TMH } &  Kruskal-Wallis & 0.93  & 3.35E-01 & 7.64E-03 & 0.10  & 7.56E-01 & 2.37E-03 & 0.84  & 3.60E-01 & 1.40E-02 \\
          &  Kolmogorov-Smirnov & 0.13  & 6.36E-01 & 3.17E-03 & 0.12  & 9.24E-01 & 6.69E-04 & 0.19  & 5.28E-01 & 8.76E-03 \\
          &  Student's T-test & -0.86 & 3.90E-01 & 6.58E-03 & 0.21  & 8.31E-01 & 1.57E-03 & 0.79  & 4.33E-01 & 1.15E-02 \\
    \multirow{3}[0]{*}{TMH and flanks } &  Kruskal-Wallis & 0.04  & 8.52E-01 & 1.12E-03 & 0.12  & 7.28E-01 & 2.69E-03 & 0.04  & 8.33E-01 & 2.51E-03 \\
          &  Kolmogorov-Smirnov & 0.11  & 7.72E-01 & 1.81E-03 & 0.13  & 8.79E-01 & 1.09E-03 & 0.11  & 9.80E-01 & 2.81E-04 \\
          &  Student's T-test & -0.22 & 8.23E-01 & 1.37E-03 & -0.38 & 7.04E-01 & 2.97E-03 & -0.19 & 8.50E-01 & 2.22E-03 \\
    \end{tabular}%
				\label{table:speciestableswissprotstats}

\end{table}%

\begin{table}[htbp]
\centering
\captionof{table}[Hydrophobicity statistical comparisons between mouse and human, yeast, and plants in the UniProt Curated Dataset.]
{\textbf{Hydrophobicity statistical comparisons between mouse and human, yeast, and plants in the UniProt Curated Dataset.}
Here, we compare a mammalian set of \gls{ta} proteins (Human N=38 and Mouse N=37) to \textit{A. thaliana} (N=60) representing plants  and  \textit{S. cerevisiae} (N=31) representing yeasts.
The hydrophobicity was predicted as the mean average of the values of the sequences of the \gls{tmh}, as well another group including up to $\pm$5 flanking residues predicting the boundary of \gls{tmh}s is difficult, according to the Kyte \& Doolittle hydrophobicity scale~\cite{Kyte1982}.
%Disorder was calculated in the same way using the GlobProt scale \cite{Linding2003}.
The Test column refers to the statistical score obtaineed from the test; H statistic for the Kruskal Wallis, the KS statistic for the Kolmogorov Smirnov test, and the t-statistic for the T-test.
$P$ is the P-value of that statistical score.
$B$ refers to the Bahadur slope, an interpretation of the P-value that accounts for the sample size powering the test~\cite{Bahadur1967, Bahadur1971}.}
	\tiny
	% Table generated by Excel2LaTeX from sheet 'SwissProt filtered species'

    \begin{tabular}{clrrrrrrrrr}
	          &       & \multicolumn{3}{c}{Mammal and Plant} & \multicolumn{3}{c}{Mammal and Yeast} & \multicolumn{3}{c}{Plant and Yeast} \\
	          &       & \multicolumn{1}{l}{ Test} & \multicolumn{1}{l}{ $P$} & \multicolumn{1}{l}{ $B$} & \multicolumn{1}{l}{ Test} & \multicolumn{1}{l}{ $P$} & \multicolumn{1}{l}{ $B$} & \multicolumn{1}{l}{ Test} & \multicolumn{1}{l}{ $P$} & \multicolumn{1}{l}{ $B$} \\
	    \multirow{3}[0]{*}{TMH } &  Kruskal-Wallis & 2.15  & 1.42E-01 & 1.45E-02 & 0.22  & 6.39E-01 & 4.22E-03 & 2.30  & 1.30E-01 & 2.25E-02 \\
	          &  Kolmogorov-Smirnov & 0.17  & 2.86E-01 & 9.27E-03 & 0.15  & 6.32E-01 & 4.34E-03 & 0.24  & 1.69E-01 & 1.95E-02 \\
	          &  Student's T-test & -1.71 & 8.96E-02 & 1.79E-02 & 0.04  & 9.70E-01 & 2.86E-04 & 1.47  & 1.46E-01 & 2.11E-02 \\
	    \multirow{3}[0]{*}{TMH and flanks } &  Kruskal-Wallis & 2.17  & 1.41E-01 & 1.45E-02 & 0.14  & 7.13E-01 & 3.19E-03 & 0.59  & 4.41E-01 & 9.00E-03 \\
	          &  Kolmogorov-Smirnov & 0.21  & 8.30E-02 & 1.84E-02 & 0.10  & 9.62E-01 & 3.62E-04 & 0.14  & 8.00E-01 & 2.45E-03 \\
	          &  Student's T-test & -1.33 & 1.86E-01 & 1.25E-02 & -0.39 & 7.00E-01 & 3.36E-03 & 0.69  & 4.90E-01 & 7.83E-03 \\
	    \end{tabular}%
					\label{table:speciestableuniprotstats}
	\end{table}%


\subsection{Organelle Membrane Variation}
% Average lines for figure, table for stats

When we consider the \gls{ta} proteins at certain locations within the cell ignoring species, we see clear differences in the biochemistry of the \gls{tmh}.
In the UniProt manually curated dataset, the Kyte \& Doolittle hydrophobicity scores rage from 1.7 in mitochondria to 2.7 in the \gls{pm} (Figure \ref{fig:average_organelle_factors_ta}A).
Similarly, in the SwissProt filtered dataset the mitochondria lies at 1.9, but it appears to be the Golgi apparatus that is the peak at 2.4.

With such a strong hydrophobic difference that, as we saw in the species, may not be solely accounted for by adaptation to the different membrane compositions, it is important to consider if a cryptic targetting signal exists within the \gls{tmh} themselves for \gls{ta} proteins.
The linguistic sequence entropy of the \gls{tmh} string as well as the disorder were also examined.
In terms of sequence entropy, there is a marked lack of entropy in the \gls{pm} subset (mean entropy = 3.12 in the \gls{tmh}, 2.6 including $\pm$5 flanking residues) from the UniProt curated dataset compared to the other organelle datasets (entropy $>$ 3.3 and $>$ 2.8 including the flanks).
However this stark difference cannot be observed in the SwissProt set.
This is unsurprising given that the hydrophobic nature of the \gls{tmh}s demands that certain residues must be overrepresented, which lowers the sequence entropy.
In this case, we have a highly hydrophobic set, the \gls{pm} UniProt set, which likely contains a higher proportion of the most hydrophobic residues.

When considering disorder, we see in both datasets that the Golgi subset is the most negative at -0.2 with \gls{tmh} and -0.25 with the flanks in both the UniProt manually curated and the SwissProt filtered datasets.
Whereas mitochondria is the least negative in both cases when considering the \gls{tmh} (-0.14 in UniProt and -0.13 in Swissprot) or the \gls{tmh} and $\pm$5 flanking residues (-0.18 in UniProt and -0.16 in Swissprot).

\begin{figure}[!ht]
\centering
\includegraphics[width=1\textwidth]{TA_chapter/organelle-averages}
\captionof{figure}[Average sequence-based biochemical values of organelle datasets from UniProt manually curated set and SwissProt automatically filtered dataset.]
{\textbf{Average sequence-based biochemical values of organelle datasets from UniProt manually curated set and SwissProt automatically filtered dataset.}

A) The average hydrophoicity values from the Kyte \& Doolittle scale~\cite{Kyte1982}, B) the average sequence entropy~\cite{Shannon1948} (see methods) and C) the GlobProt scale~\cite{Linding2003} for both the \gls{tmh} and the \gls{tmh}$\pm$5 residues.
Values are shown for both the UniProt manually curated set and the SwissProt filtered set.
In the UniProt manually curated set we compare \gls{ta} proteins from the~\gls{er} (N=400) to the Golgi (N=82), the~\gls{pm} (N=37), and the mitochondria (N=401).
For the SwissProt filtered set we compare \gls{ta} proteins from the~\gls{er} (N=98) to the Golgi (N=82), the~\gls{pm} (N=157), and the mitochondria (N=65).
Error bars are shown at $\pm 1 \sigma$ from the mean of the respective dataset.
}

\label{fig:average_organelle_factors_ta}
\end{figure}

%\subsection{SNAREs are consistently more hydrophobic across the entire \gls{tmh}}
% Average lines for figure, table for stats

%~\gls{snare} proteins were noted to have more of the most hydrophobic amino acids than the general population of tail anchors~\cite{Kalbfleisch2007}.
%We exploit the difference by mapping the hydrophobicity of the transmembrane domains from a novel list of potential~\gls{ta} proteins generated in this study onto the experimentally validated~\gls{ta} hydrophobicity plot compiled by Kalbfleisch et al. published in Traffic 2007 (8: 1687-1694).
%This method has revealed potential~\gls{ta}~\gls{snare}s with~\gls{snare} motif domains that may not appear in conventional screening.

\subsection{Spontaneous insertion may be achieved by polar patches in the \gls{tmh}}
%Spont TA lines against background dataset.

%In addition, predicted insertion machinery dependent TAs with TMHs that are more polar on average than spontaneously inserting~\gls{ta} protein cytochrome b5 have been highlighted.

%move on to case studies involving spontaneous insertion?


%Our experimental collaborators in Stephen High’s group are interested in a small group of tail-anchored proteins that have very polar trans-membrane domains and are capable of liposome membrane insertion without insertion machinery, also known as spontaneous insertion.
%They have found that chimeric synaptobrevin, one of the first identified \gls{snare} proteins, is capable of spontaneous insertion if the tail anchor domain is replaced by the \gls{tm} domains belonging to a protein of known spontaneously inserting domains.
%Their studies have moved the focus of spontaneous insertion away from the loop regions and onto the physicochemical factors of the \gls{tmh} itself.


\section{Discussion}

Given the large biochemical distinction between \gls{ta} proteins with different terminal destinations, it is tempting to conclude that \gls{tmh}s contain the necessary biological factors to determine their targetting.
It is indeed possible that our observations are adaptations to the membrane environment, and this would not be unreasonable, except that \gls{ta} proteins would be expected to experience similar adaptations at a species level, for which at this sample size such an effect is unobservable.
Here, we postulate that there is indeed biological information held within the \gls{tmh} itself that allows the protein to be specifically targetted to a destiantion.
This is almost certainly aided by other factors and is part of a system with several redundant mechanisms.

This could indeed be a cryptic functional similarity to the signal anchored proteins.
Signal anchored proteins, proteins that contain a single hydrophobic segment that serves as both a mitochondrial targeting signal and a membrane anchor, as well as tail-anchored proteins have been shown to be able to spontaneously insert into the membrane independently from the translocon~\cite{Elisa2012, Lan2000, Colombo2009}.

The idea that \gls{snare} proteins are modular and capable of spontaneous insertion has significant implications for both biomedical application in liposome-based drug delivery and can aid future research for testing complex biological molecular networks~\cite{Allen2013, Nordlund2014}.

\section{Methods}

\subsection{Building a List of Tail-Anchors}
Steps carried out by Kalbfleisch \textit{et al.} published in Traffic 2007 (8: 1687\-1694)~\cite{Kalbfleisch2007}, were recreated using up to date tools.
Whilst their study focussed on the human proteome, here we take into account the entire TrEMBL and Swiss-Prot database and then stratify the datasets by the organism at the end of the pipeline.

\subsubsection{Swiss-Prot Tail Anchored Dataset According to Filters}
There were 557012 protein records downloaded from Swiss-Prot via UniProt~\cite{TheUniProtConsortium2014} (Downloaded 24--04--2018).
106149~\gls{tmh}s (\url{TRANSMEM} annotation) were found between 76953 records (\url{annotation:(type:transmem) AND reviewed:no}).
This keyword is contained in a record according to either experimental evidence~\cite{TheUniProtConsortium2014} or a robust meta-analysis of~\gls{tmh} prediction using TMHMM~\cite{Krogh2001}, Memsat~\cite{Jones2007}, Phobius~\cite{Kall2004,Kall2007} and the hydrophobic moment plot method of Eisenberg and co-workers~\cite{Eisenberg1984}.
11141 of those records had only a single~\gls{tmh}.
11110 of those~gls{tmh}s were within the length thresholds of 16 to 30 residues (None of those had the annotation for splice isoforms according to \url{NON_TER} annotation).
5548 of those had had no~\gls{sp} annotation (\url{SIGNAL}).
4332 of those had annotation (based on \url{TOPO_DOM} annotation) that the N terminal was cytoplasmic.
615 of those had the~\gls{tmh} within 25 residues of the C terminal, the same threshold used by Kalbfleisch and their coworkers~\cite{Kalbfleisch2007}.
Running CD-Hit 4.5.3 on the WebMGA webserver~\cite{Huang2010, Wu2011} at 90\% identical sequence at 90\% coverage thresholds resulted in 443 representative proteins.
This threshold was chosen as a compromise between avoiding over-representation of a certain protein and maintaining a viable sample size.

From this representative list, 46 were Archaeal, 66 were bacterial, and 320 were Eukaryotic and 11 came from dsDNA viruses.
When counting proteomes with greater than 20 records, 49 belonged to the \textit{A. thaliana} proteome, 48 to Mouse, 46 to the human proteome, 24 to \textit{S.cerevisiae}. %19 from RAT!

65 were annotated under the Mitochondrion location (query \url{locations:(location:"Mitochondrion [SL-0173]")}), 157 in the \gls{pm} (query \url{locations:(location:"Cell membrane [SL-0039]")}, 82 in the Golgi (query \url{locations:(location:"Golgi apparatus [SL-0132]")}), and 98 in the \gls{er} (query \url{locations:(location:"Endoplasmic reticulum [SL-0095]"}).

\subsubsection{TrEMBL Tail Anchored Dataset According to Filters}
111425234 records were storeed in the TrEMBL database at time of download (Downloaded 25--04--2018).
22107826 of those contained \url{TRANSMEM} annotation (\url{annotation:(type:transmem) AND reviewed:no}).
18053 of these were single-pass proteins.
All of these were within the length restrictions.
17973 of those did not contain a signal sequence when looking for \url{SIGNAL} annotation.
5157 of those contained a cytoplasmically located N terminal according to \url{TOPO_DOM} annotation.
155 records had a~\gls{tmh} within 15 residues of the C terminal residue.
In those record's annotations, no more than 1 appeared in any given species, so they were omited from the Swiss-Prot list sequence redundancy protocol to avoid representing a well annotated record with a poorly annotated record.

\subsubsection{UniProt Curated List}
A query for \url{locations:(location:"Single-pass type IV membrane protein [SL-9908]")} was used in UniProt which returned 2460 UniProtKB IDs; 463 Swiss-Prot results and 1997 TrEMBL results.
Running these records through CD-HIT at 90\% redundancy yielded 309 Swiss-Prot records and 808 TrEMBL records~\cite{Huang2010, Wu2011}.
Of those, 987 proteins from 973 records (308 from Swiss-Prot, and 665 from TrEMBL) had the \url{TRANSMEM} annotation indicating a bone fide~\gls{tmh}.
No further filters were applied to this list.
Proteomes represented by more than 20 records include \textit{A. thaliana} (60 records), Humans (38), Mouse (37), and \textit{S. cereveisiae} (31). % and 20 to Rat.

401 were annotated under the Mitochondrion location (query \url{locations:(location:"Mitochondrion [SL-0173]")}) 39 from Swissprot and 362 automatically assigned in TrEMBL.
401 in the \gls{er} (query \url{locations:(location:"Endoplasmic reticulum [SL-0095]"}), 98 from SwissProt and 303 automatically annotated in TrEMBL.
1 TrEMBL record (A0A1E5RT24) in the \gls{er} set contained an ``X'' residue in the C terminal flank and was omitted from the analyses.
Two subcellular location datasets had no automatically ascribed records and only contained manually annoted SwissProt records; 37 in the \gls{pm} (query \url{locations:(location:"Cell membrane [SL-0039]")}, and 82 in the Golgi (query \url{locations:(location:"Golgi apparatus [SL-0132]")}).

\subsubsection{Remapping Previous Dataset}
189 of the 411 proteins from the previous study~\cite{Kalbfleisch2007} were successfully mapped to 222 UniProtKB IDs using the UniProt mapping tools with the RefSeq Protein to UniProtKB option~\cite{TheUniProtConsortium2014}.

\subsection{Calculating Hydrophobicity}
Windowed hydrophobicity was calculated using a window length of 5 residues, and half windows were permitted.
Average hydrophobicity takes the total of the raw amino acid hydrophobicity values and divides them by the number of amino acids in the slice.
Values reported in the results are based on the Kyte \& Doolittle scale~\cite{Kyte1982} which is based on the water\---vapour transfer free energy and the interior-exterior distribution of individual amino acids.
%Hydrophobicity values were also validated by the White and Wimley scale~\cite{White1999}, the Hessa scale~\cite{Hessa2005}, and the Eisenberg scale~\cite{Eisenberg1984}.

\subsection{Calculating Sequence Entropy}
Sequence entropy, is essentially an estimate of the linguistic entropy of a string.
In the context of biology, it can be thought of as an estimation of the non-randomness of a sequence.
Sequence complexity can be used to analyse DNA sequences~\cite{Pinho2013, Oliver1993, Troyanskaya2002}, and is a component of the TMSOC z-score which can predict function beyond anchorage of a \gls{tmh}~\cite{Wong2011, Wong2012, Baker2017}.
here we focus on the analysis of the complexity of a sequence in protein sequences.

Broadly speaking, the information theory entropy of a linguistic string can be defined as in equation~\ref{simpleentropy2}, and we treat the protein sequence \gls{tmh} as a string with or without its flanking regions.

\begin{equation} \label{simpleentropy2}
	H(S)=-{\sum_{i=1}^n {p_i\log_s(p_i)}}
\end{equation}

Where H is the entropy of a sequence (S), and $p_i$ is the probability of a character $i$ through each position (n) in S. This allows us to quantify the average relative information density held within a string of information~\cite{Shannon1948}.

\subsection{Statistics}

The null hypothesis of homogeneity of two distributions was examined with the~\gls{ks}, the~\gls{kw} and the 2-sampled T-test statistical tests.
These tests were all ran through the Python scipy stat package v0.17 python package~\cite{VanderWalt2011}.
To note, the~\gls{ks} test scrutinises for significant maximal absolute differences between distribution curves; the~\gls{kw} test is after skews between distributions and the student t-test statistical test checks the average difference between distributions.

Since the ``P‑value'' is a product of a fraction of a permutated set that exponentially increases as N increases, the P-value is a strong function of N.
We rely on the (absolute) Bahadur slope ($B$) as a measure of distance between two distributions~\cite{Bahadur1967, Bahadur1971, Sunyaev1998, Baker2017}:

\begin{equation} \label{eq:bahadur2}
B=\frac{|\ln(P~value)|}{N}
\end{equation}

The larger the absolute Bahadur slope, the greater the difference between the two distributions.


\chapter{Co-operative TMHs}
\section{Abstract}
\section{Introduction}

Translocation is when a ribosome translates the~\gls{rna} to a nascent peptide chain which is handed directly or indirectly to the translocon insertion machinery which threads the chain through and, in the case of~\gls{tmh}s, releases the~\gls{tmh} into the membrane environment.

The overwhelming majority of~\gls{tmp}s use the co-translational method of translocation.
It has long been understood that this method is essentially the ~\gls{srp} recognising and attaching to the nascent peptide chain whilst it is still associated with the ribosome, and the~\gls{srp} then targets the peptide and ribosome to a~\gls{sr} in association with some membrane insertion machinery on the~\gls{er} membrane~\cite{Pool2005, Hessa2005}.

Crystal structures showed the \gls{srp} targets the nascent peptide chain for membrane insertion via a GTPase in both the \gls{srp} and \gls{sr}, that is initially associated with the translocon machinery, coming together to form a complex thus bringing the nascent peptide chain in proximity to the translocon~\cite{Shan2005}.
Mutant studies of \gls{srp} revealed key discrete conformational stages~\cite{Shan2005}.
These are the specific recognition of signal sequences on cargo proteins, the targeting of the package to the membrane, the handing over of the cargo to the translocation machinery all the while maintaining precise spatial and temporal coordination of each molecular event \cite{Saraogi2011}.

%Section on translocon

The prevailing idea about membrane insertion by the translocon is that the \gls{tmh}s partition in the membrane one at a time as the translocon lateral gate opens, exposing the \gls{tmh} to the membrane (Figure \ref{fig:sequential-insertion})\cite{Cymer2014}.

\begin{figure}[!ht]
\centering
\includegraphics[width=1\textwidth]{multipass-folding/sequential-insertion}
        \captionof{figure}[A cartoon showing the generally accepted schematic of sequential multipass TMH insertion into the membranes.]{\textbf{A cartoon showing the generally accepted schematic of sequential multipass TMH insertion into the membranes.}
        The two key concepts are that, one at a time, the TMHs emerge from the ribosome into the translocon.
        This triggers the lateral gate to open.
        As the nascent TMH is exposed to the membrane, it begins to partition.
        This implies that the TMHs ultimately have no meaningful interactions with one another until they the protein has been threaded into the membrane and the1 TMH bundle is formed.
}
\label{fig:sequential-insertion}
\end{figure}

\subsection{Co-operative insertion}
Generally only a handful of examples have come close to demonstrating that \gls{tmh}s co-operate in order to integrate relatively polar \gls{tmh}s.

\subsubsection{GPCRs}

\gls{gpcr}s are a diverse family of membrane surface receptors with 7 \gls{tmh} segments.
They have adapted to respond to a wide range of specific signals ranging from macromolecules, to photons.
The specific signal triggers a conformational change of the \gls{gpcr} that is translated across the membrane.
\gls{gpcr}s have been associated with tumerigenesis \cite{OHayre2013}, metastasis \cite{Singh2015} and in cancers \cite{Bar-Shavit2016} and are a potential target for therapies \cite{Arakaki2018}.
Their ubiquitous presence in cellular life and medical relevance makes them an important topic of study.

Opsins are a group of light sensitive \gls{gpcr}s.
It was shown by cross linking studies that opsin \gls{tmh}s 5-7 are retained in the \gls{er} translocon and only parition once biosynthesis is complete \cite{Ismail2008}.
The timing of this partitioning is controlled by the hydrophobicity of the \gls{tmh}, not protein length or the relative position of the \gls{tmh} within the protein.
Although artificially extending the C-terminal did not result release of the \gls{tmh}s, by replacing native \gls{tmh} 7 with a more hydrophobic \gls{tmh}, the speed of insertion was decreased.
\gls{tmh}s 1-4 are inserted independently, and the 5-7 \gls{tmh}s partition into the membrane at the same time.

\subsection{Voltage gated ion channels}

Good\--bad graphic from von heijne group association can happen in tunnel, in translocon, or after insertion Cymer, F., von  Heijne, G., \& White, S. H. (2015). Mechanisms of Integral Membrane Protein Insertion and Folding. Journal of Molecular Biology, 427(5), 999–1022. https://doi.org/10.1016/J.JMB.2014.09.014

Potassium channel (shaker family) S3–S4 insertion may, in part, occur in a sequential or “cotranslational” manner potassium channel - Zhang,L., Sato,Y., Hessa,T., von Heijne,G., Lee,J.-K., Kodama,I., Sakaguchi,M. and Uozumi,N. (2007) Contribution of hydrophobic and electrostatic interactions to the membrane integration of the Shaker K+ channel voltage sensor domain. Proc. Natl. Acad. Sci. U. S. A., 104, 8263–8.

SecY “cracks” open according to crystal structure analysis. Allows potential TMH to probe the environment. Egea,P.F. and Stroud,R.M. Lateral opening of a translocon upon entry of protein suggests the mechanism of insertion into membranes. 10.1073/pnas.1012556107.

Translational Arrest Peptides as in Vivo Force Sensors found interactions between C-terminal TMH and upstream TMH as the C-terminal TMH is partitioning from the translocon. 1. Cymer,F. and von Heijne,G. (2013) Cotranslational folding of membrane proteins probed by arrest-peptide-mediated force measurements. Proc. Natl. Acad. Sci. U. S. A., 110, 14640–5.

\subsection{Ribsomes in the biogenesis of membrane proteins.}
Ribosomes translate mRNA sequences to amino acid chains and are present in all living cells, and indeed the ribosomal complexes presence and activity is for many used to define whether something is alive.
They are a highly conserved RNA-protein complex with a multitude of accessory proteins and targetting factors.

During translation of a \gls{tmp} protein, the \gls{srp} binds to the ribosome after recognising the nascent protein as a \gls{tmp}.
This complex then binds to the \gls{sr} in association with the membrane bound translocon.
The nascent peptide is then fed into the translocon as it is being translated; hence ``co-translational insertion''.

The journey of the \gls{tmh} through this machinery has been studied using both crosslinking experiments and a relatively new technique of \gls{ap}s \cite{Cymer2014}.
Accessibility assays and an improved intramolecular crosslinking assay showed that the helical transmembrane S3b–S4 hairpin (the “paddle”) of a voltage-gated potassium (Kv) forms in the ribosome tunnel \cite{Tu2014}.
Ribosomal folding of the \gls{tmh}s in Kv1.3, a potassium channel, is maintained in the translocon \cite{Tu2010}.
Therefore, the tertiary folding of the voltage sensor domain occurs via preformed secondary-structure formation.

Furthermore, it has recently been suggested that larger structures fold as the ribosomal exit tunnel widens \cite{Kudva2018}.
This size dependent folding was observed by using the SecM translational \gls{ap}.
Two ribosome mutants were compared (uL23 that is close to the exit tunnel and uL24 deletions which is a hairpin loop that obstructs the tunnel exit.) zinc finger folds deeper in uL23 mutant than wildtype (but not uL24) and a 100 residue domain folds deeper than the uL24 mutant (but not the uL23) \cite{Kudva2018}.

The ribosomal tunnel also speeds up elongation of neutral and negatively-charged peptides.
This is attributed to the sporadic negative patches within the ribosomal exit tunnel \cite{Lu2008}.

The ribosome clearly has the potential to prefold motifs and small domains before translocon insertion.

\section{Methods}
\subsection{Datasets}
\subsection{Complexity}
\subsection{Statistics}

\section{Results}
\subsection{There are step changes in TMH complexity depending on the TMH number in GPCRs}
GPCR distribution tables for complexity and hydrophobicity

Graphs of complexity and hydrophobicity distributions

Show there are step changes in GPCRs from Bahadur

Supplementary tables for additional stats tests and hydrophobicities

\subsection{Complexity ascention repeats according to how many TM-bundles are in the protein.}
GPCR distribution tables for complexity and hydrophobicity
Graphs of complexity and hydrophobicity distributions
Show there are step changes in GPCRs from Bahadur
Supplementary tables for additional stats tests and hydrophobicities

\subsection{The pattern is present for GPCR subfamilies}
Figure of complexity distributions with Rhodopsin like, Secretin, metabotropic glutamate, Fungal mating, cyclic AMP, Frizzled and smooth.
Bahadur tables also

\subsection{The prevelance of this amongst all TMPs.}
Mechano-sensitive (controlled vocabulary if no list available) distributions
Voltage gated (controlled vocabulary if no list available) distributions

\section{Discussion}
GPCRs have long be known to be overrepresented among genomes \cite{Remm2000}.

Seen across a variety of 7TM families with varying functions with datasets built from all membrane types (hence variety).
Suggests a pressure for simpler TMHs to precede more complex ones, repeating every 3-4 TMHs.
The universality points toward translocon behaviour pressure, or thermodynamic stability in the membrane.
Would we expect this behaviour if the translocon acted on only one TMH at a time?


\chapter{Conclusions and outlook}
%The description of a~\gls{tmh} remains incomplete.
%The understanding of~\gls{tmp} topology is erroneous, and despite a wealth of structures, the general model of helix-helix and helix-lipid interactions remains speculative and requires a great deal of intensive analysis to generate a working model of a particular~\gls{tmp}.
\sloppy
As the idea of positive residues inside the cytoplasm emerged during the late 1980s, so did the idea of negative residues working in concert with~\gls{tmh} orientation.
It was shown that removing a single lysine residue reversed the topology of a model \textit{erichia coli} protein, whereas much higher numbers of negatively charged residues are needed to reverse topology~\cite{Nilsson1990}.
One would also expect to see a skew in negatively charged distribution if a cooperation between oppositely charged residues orientated a~\gls{tmh}, however there is no conclusive evidence in the literature for an opposing negatively charged skew~\cite{Granseth2005, Nilsson2005a, Sharpe2010, Baeza-Delgado2013, Pogozheva2013}.
However, in \textit{E.
coli} negative residues do experience electrical pulling forces when travelling through the SecYEG translocon indicating that negative charges are biologically relevant~\cite{Ismail2015}.
In this chapter, we explore the literature surrounding charged residue distribution in the~\gls{tmh}, and demonstrate that the ``negative-outside'' skew exists in anchoring~\gls{tmh}s
In this thesis we ellucidated the ``negative\--outside'' rule.
Additionally we have made observations regarding how anchoring \gls{tmh}s are optimised to their membranes.
This new insight into the bahviour of \gls{tmh}s will not only help us understand specific case in biology, but can also inform synthetic biology regarding \gls{tmp}s.

With respect to \gls{ta} \gls{tmp}s, we found differences between the mitochondrial and secretory dedtermined \gls{tmh}s.
We found alanine is used more commonly in the mitochondrial \gls{tmh}s, perhaps even moreso than isoleucine and leucine.
We also observed a reversal of the positive\--inside negative\--outside distributions of charged residues.
This re\--emphasises the notion that there is more to \gls{ta} protein trafficking than the hydrophobicity of the \gls{tmh} alone, and could inform more specific studies of individual proteins.
%\gls{ta} proteins in drug delivery
However, in order to reach more thorough conclusions at a molecular level, more annotation is needed in the databases regarding chaperone interactions, and a more precise terminology of what constitutes a post translationally inserted \gls{ta} protein.

Understanding cooperative \gls{tmh} insertion and the integration of marginally hydrophobic \gls{tmh}s will require a much more thorough understanding of the temporal aspects of the process. %EM and appears
However, understanding the relationship \gls{tmh}s have with one another is critical for developing complex synthetic \gls{tmp}s.


% In order for this to work a special build sequence is needed. http://tex.stackexchange.com/a/46732/42423

%This line prints the bibliography ;)
%One of the references has a "-" in the wrong place and screws up the Bibliography. %the planque reference is messed up. It may need changing with every bib.tex update unless the permanent record is changed.
\clearpage
\printbibliography[title={Bibliography}]

% Uncomment the following THREE lines if you do have an Appendix
%\appendix
%\chapter{}
%.........

\end{document}
