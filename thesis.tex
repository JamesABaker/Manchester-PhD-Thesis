% The 12pt option is required by the 2001/02 thesis regulations
% Hacked together from the LaTeX template from the maths department of UoM


%This is just the standard line that tells the virtual "printer" what document we're making.
% Remove the twoside option for single-sided printing
\documentclass[12pt,PhD,twoside]{muthesis}

%These are the citation styles.
\usepackage[backend=bibtex,style=nature,citestyle=nature]{biblatex} %science, nature, alphabetic are common choices.
%FYI


% Citations styles = https://www.sharelatex.com/learn/Biblatex_citation_styles
% Bibliography styles = https://www.sharelatex.com/learn/Biblatex_bibliography_styles


\usepackage{graphicx} %use [!ht] (which means ignore sensible protocols and place here) not [h] (which means place here-ish) for figures otherwise they end up at the end of the chapter.

\usepackage{amsmath}
\usepackage{csquotes}
\usepackage{longtable}
\usepackage[acronym]{glossaries} %Whilst it's tempting, I've found tinkering with glossary structure and style is a nightmare. Avoid at all costs.
\addbibresource{references.bib} % This is the folder containing all your "bibtex" references. All reference managers can export to this format.
\graphicspath{{images/}} %Store all your figures here

% abbreviations: The first option in the curly braces is the code used in the text, second is the short form, and third is the long form. http://tex.stackexchange.com/questions/86666/how-to-create-both-list-of-abbreviations-and-list-of-nomenclature-using-nomencl

% abbreviations: 1.the code used in the text, 2. the short form, 3. the long form. http://tex.stackexchange.com/questions/86666/how-to-create-both-list-of-abbreviations-and-list-of-nomenclature-using-nomencl
\newacronym{pm}{PM}{Plasma Membrane}
\newacronym{md}{MD}{Molecular Dynamics}
\newacronym{tm}{TM}{Transmembrane}
\newacronym{tmh}{TMH}{Transmembrane Helix}
\newacronym{tmp}{TMP}{Transmembrane Protein}
\newacronym{er}{ER}{Endoplasmic Reticulum}
\newacronym{ta}{TA}{Tail Anchor}
\newacronym{pdb}{PDB}{Protein Data Bank}
\newacronym{snare}{SNARE}{Soluble N-Ethylmaleimide-Sensitive Factor Attachment Receptor}

% nomenclature:
%\newglossaryentry{angelsperarea}{
  %name = $a$ ,
%  description = The number of angels per unit area,
%}

\makeglossaries

\begin{document}
\title{Investigating the Recognition and Interactions of Non-Polar $\alpha$ Helices in Biology}
\author{James Baker}
\faculty{Life Sciences}
\def\wordcount{xxxxx}

% Uncomment the line below to suppress the `List of Tables' page (optional)
\tablespagefalse

% Uncomment the line below to suppress the `List of Figures' page (optional)
\figurespagefalse

% Uncomment the line below to use a customised Declaration statement
%\def\declaration{All the work in this thesis has been sourced from Google}

\beforeabstract % Don't move the abstract, it doesn't like it.
Transmembrane $\alpha$ helix containing proteins make up around a quarter of all proteins, as well as two thirds of drug targets, and contain some of the most critical proteins required for life as we know it. Yet they are fundamentally difficult to study experimentally. This is in part due to the very features that make them so biologically influential: their hydrophobic transmembrane helices. What is missing in the current literature is a nuanced understanding of the complexities of the helix composition beyond a hydrophobic region of around 20 residues. Currently it is known that the properties of transmembrane protein $\alpha$ helices underpin membrane protein insertion mechanisms and furthermore can be used to predict presence of function in the transmembrane helix itself. By leveraging large datasets of transmembrane proteins, this thesis is focussed on characterising features of $\alpha$ helices en masse, particularly regarding their topology, membrane-protein interactions, and intra-membrane protein interactions.

Herein we expand on the core understanding of the biophysicochemical properties of these helices. We find evidence of a universal ``negative-not-inside'' rule that complements the famous ``positive-inside rule'' as well as intramembrane leucine propensity for the inner leaflet.

Furthermore we provide an up-to-date dataset of potential Tail-Anchored proteins, a group of post-translationally inserted proteins.
\afterabstract


%The preface section doesn't like being in a different document for some reason. (I think it's just my compiler, there is no reason why it shouldn't work)
\prefacesection{Acknowledgements}
Together my supervisory team have instilled in me an excitement of discovery. Furthermore they have taught me a deep value of inductive reasoning and inference over deduction.

Shout out to all the Biopython devs that saved my sanity!

\subsection*{Preamble} %A side note. the "*" hides the section number.
In the 1950s and 1960s, the field of biological philosophy was still emerging. David Hull writes on the matter in his 1969 entitled {\it``What Philosophy of Biology is not''}:

\begin{displayquote}
``Periodically through the history of biology, biologists have tried to do
a little philosophy...'' \cite{Hull1969}
\end{displayquote}

I think this accurately summarises my attempts at applying philosophy to science for my philosophical doctorate. I can't help but wish my thesis title was ``The ins-and-outs of greasy peptides''.

So long, and thanks for all the fish!

\prefacesection{List of publications}

\subsection*{Journal Articles} % the * hides the section number

\subsection*{Posters}
Baker, J. and Warwicker, J. A Bioinformatic Method to Identify Potential SNARE Proteins. {\it 40th FEBS Congress} Late Breaker (2015)

\afterpreface % DO NOT DELETE. BAD THINGS HAPPEN.

%This is where we call the chapters. No need to include ".tex" The chapters will be printed to the document in this order.
\input{chapterIntroduction}
\input{chapter02}
\chapter{Tail-anchored protein discovery} %Mutants outlined + any results from Abbi
\section{Introduction}
\section{Methods}
\section{An up to date tail-anchor dataset}
\section{Potential tail-anchored SNARE protein discovery}
\section{Investigating biology of spontaneously inserting tail anchored proteins}

\input{chapter04}
\input{chapterConclusions}

% This would be better at the begining so that before we get started, let's print the glossaries.
% In order for this to work a special build sequence is needed. http://tex.stackexchange.com/a/46732/42423
\printglossary[type=\acronymtype,title=Abbreviations]
\printglossary[title=Nomenclature] % Uncomment this line to use the nomenclature section.

%This line prints the bibliography ;)
%One of the references has a "-" in the wrong place and screws up the Bibliography. %the planque reference is messed up. It may need changing with every bib.tex update unless the permannet record is changed.
\printbibliography[title={Bibliography}]

% Uncomment the following THREE lines if you do have an Appendix
%\appendix
%\chapter{}
%.........

\end{document}
