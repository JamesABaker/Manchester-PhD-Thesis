
\chapter{The ``Negative-Outside'' Rule}

The description of a \gls{tmh} remains incomplete. The understanding of \gls{tmp} topology is erroneous, and despite a wealth of structures, the general model of helix-helix and helix-lipid interactions remains speculative and requires a great deal of intensive analysis to generate a working model of a particular \gls{tmp}.

The work presented in this chapter is an expanded version of published work~\cite{Baker2017}. We use advanced statistical analysis to analyze large sequence datasets that have rich topological annotation. By analyzing these sequences in the context of anchorage, we find that some \gls{tmh}s are confined to biological constraints of the membrane, whereas others that likely contain function beyond anchorage, are less conforming to the membrane. Specifically, there is further elaboration of statistical definitions in the methods than in the published paper.



\section{Abstract}

\section{Summary}
As the idea of positive residues inside the cytoplasm emerged during the late 1980s, so did the idea of negative residues working in concert with \gls{tmh} orientation. It was shown that removing a single lysine residue reversed the topology of a model \textbf{\textit{Escherichia coli}} protein, whereas much higher numbers of negatively charged residues are needed to reverse topology~\cite{Nilsson1990}. One would also expect to see a skew in negatively charged distribution if a cooperation between oppositely charged residues orientated a \gls{tmh}, however there is no conclusive evidence in the literature for an opposing negatively charged skew~\cite{Granseth2005, Nilsson2005, Sharpe2010, Baeza-Delgado2013, Pogozheva2013}. However, in \textbf{\textit{E. coli}} negative residues do experience electrical pulling forces when traveling through the SecYEG translocon indicating that negative charges are biologically relevant~\cite{Ismail2015}.
\section{Methods}

\subsection{Normalisation}

\begin{equation}
c_r=\frac{(a_{K,r}+a_{R,r})-(a_{D,r}+a_{E,r})}{N}
\end{equation}


\begin{equation}
  p_{i,r}=\frac{a_{i,r}}{\underset{r}{\max}{(a_r)}}
\end{equation}

\begin{equation}
  q_{i,r}=\frac{{100}\cdot{a_{i,r}}}{a_i}
\end{equation}

\section{Results}
\subsection{Biophysicochemical differences in multi-pass and single-pass helices}
