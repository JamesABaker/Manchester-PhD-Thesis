
\chapter{Co-operative TMHs}
\section{Abstract}
\section{Introduction}

Translocation is when a ribosome translates the~\gls{rna} to a nascent peptide chain which is handed directly or indirectly to the translocon insertion machinery which threads the chain through and, in the case of~\gls{tmh}s, releases the~\gls{tmh} into the membrane environment.

The overwhelming majority of~\gls{tmp}s use the co-translational method of translocation.
It has long been understood that this method is essentially the ~\gls{srp} recognising and attaching to the nascent peptide chain whilst it is still associated with the ribosome, and the~\gls{srp} then targets the peptide and ribosome to a~\gls{sr} in association with some membrane insertion machinery on the~\gls{er} membrane~\cite{Pool2005, Hessa2005}.

Crystal structures showed the \gls{srp} targets the nascent peptide chain for membrane insertion via a GTPase in both the \gls{srp} and \gls{sr}, that is initially associated with the translocon machinery, coming together to form a complex thus bringing the nascent peptide chain in proximity to the translocon~\cite{Shan2005}.
Mutant studies of \gls{srp} revealed key discrete conformational stages~\cite{Shan2005}.
These are the specific recognition of signal sequences on cargo proteins, the targeting of the package to the membrane, the handing over of the cargo to the translocation machinery all the while maintaining precise spatial and temporal coordination of each molecular event \cite{Saraogi2011}.

%Section on translocon

The prevailing idea about membrane insertion by the translocon is that the \gls{tmh}s partition in the membrane one at a time as the translocon lateral gate opens, exposing the \gls{tmh} to the membrane (Figure \ref{fig:sequential-insertion})\cite{Cymer2014}.

\begin{figure}[!ht]
\centering
\includegraphics[width=1\textwidth]{multipass-folding/sequential-insertion}
        \captionof{figure}[A cartoon showing the generally accepted schematic of sequential multipass TMH insertion into the membranes.]{\textbf{A cartoon showing the generally accepted schematic of sequential multipass TMH insertion into the membranes.}
        The two key concepts are that, one at a time, the TMHs emerge from the ribosome into the translocon.
        This triggers the lateral gate to open.
        As the nascent TMH is exposed to the membrane, it begins to partition.
        This implies that the TMHs ultimately have no meaningful interactions with one another until they the protein has been threaded into the membrane and the1 TMH bundle is formed.
}
\label{fig:sequential-insertion}
\end{figure}

\subsection{Co-operative insertion}
Generally only a handful of examples have come close to demonstrating that \gls{tmh}s co-operate in order to integrate relatively polar \gls{tmh}s.

\subsubsection{GPCRs}

\gls{gpcr}s are a diverse family of membrane surface receptors with 7 \gls{tmh} segments.
They have adapted to respond to a wide range of specific signals ranging from macromolecules, to photons.
The specific signal triggers a conformational change of the \gls{gpcr} that is translated across the membrane.
\gls{gpcr}s have been associated with tumerigenesis \cite{OHayre2013}, metastasis \cite{Singh2015} and in cancers \cite{Bar-Shavit2016} and are a potential target for therapies \cite{Arakaki2018}.
Their ubiquitous presence in cellular life and medical relevance makes them an important topic of study.

Opsins are a group of light sensitive \gls{gpcr}s.
It was shown by cross linking studies that opsin \gls{tmh}s 5-7 are retained in the \gls{er} translocon and only parition once biosynthesis is complete \cite{Ismail2008}.
The timing of this partitioning is controlled by the hydrophobicity of the \gls{tmh}, not protein length or the relative position of the \gls{tmh} within the protein.
Although artificially extending the C-terminal did not result release of the \gls{tmh}s, by replacing native \gls{tmh} 7 with a more hydrophobic \gls{tmh}, the speed of insertion was decreased.
\gls{tmh}s 1-4 are inserted independently, and the 5-7 \gls{tmh}s partition into the membrane at the same time.

\subsection{Voltage gated ion channels}

Good\--bad graphic from von heijne group association can happen in tunnel, in translocon, or after insertion Cymer, F., von  Heijne, G., \& White, S. H. (2015). Mechanisms of Integral Membrane Protein Insertion and Folding. Journal of Molecular Biology, 427(5), 999–1022. https://doi.org/10.1016/J.JMB.2014.09.014

Potassium channel (shaker family) S3–S4 insertion may, in part, occur in a sequential or “cotranslational” manner potassium channel - Zhang,L., Sato,Y., Hessa,T., von Heijne,G., Lee,J.-K., Kodama,I., Sakaguchi,M. and Uozumi,N. (2007) Contribution of hydrophobic and electrostatic interactions to the membrane integration of the Shaker K+ channel voltage sensor domain. Proc. Natl. Acad. Sci. U. S. A., 104, 8263–8.

SecY “cracks” open according to crystal structure analysis. Allows potential TMH to probe the environment. Egea,P.F. and Stroud,R.M. Lateral opening of a translocon upon entry of protein suggests the mechanism of insertion into membranes. 10.1073/pnas.1012556107.

Translational Arrest Peptides as in Vivo Force Sensors found interactions between C-terminal TMH and upstream TMH as the C-terminal TMH is partitioning from the translocon. 1. Cymer,F. and von Heijne,G. (2013) Cotranslational folding of membrane proteins probed by arrest-peptide-mediated force measurements. Proc. Natl. Acad. Sci. U. S. A., 110, 14640–5.

\subsection{Ribsomes in the biogenesis of membrane proteins.}
Ribosomes translate mRNA sequences to amino acid chains and are present in all living cells, and indeed the ribosomal complexes presence and activity is for many used to define whether something is alive.
They are a highly conserved RNA-protein complex with a multitude of accessory proteins and targetting factors.

During translation of a \gls{tmp} protein, the \gls{srp} binds to the ribosome after recognising the nascent protein as a \gls{tmp}.
This complex then binds to the \gls{sr} in association with the membrane bound translocon.
The nascent peptide is then fed into the translocon as it is being translated; hence ``co-translational insertion''.

The journey of the \gls{tmh} through this machinery has been studied using both crosslinking experiments and a relatively new technique of \gls{ap}s \cite{Cymer2014}.
Accessibility assays and an improved intramolecular crosslinking assay showed that the helical transmembrane S3b–S4 hairpin (the “paddle”) of a voltage-gated potassium (Kv) forms in the ribosome tunnel \cite{Tu2014}.
Ribosomal folding of the \gls{tmh}s in Kv1.3, a potassium channel, is maintained in the translocon \cite{Tu2010}.
Therefore, the tertiary folding of the voltage sensor domain occurs via preformed secondary-structure formation.

Furthermore, it has recently been suggested that larger structures fold as the ribosomal exit tunnel widens \cite{Kudva2018}.
This size dependent folding was observed by using the SecM translational \gls{ap}.
Two ribosome mutants were compared (uL23 that is close to the exit tunnel and uL24 deletions which is a hairpin loop that obstructs the tunnel exit.) zinc finger folds deeper in uL23 mutant than wildtype (but not uL24) and a 100 residue domain folds deeper than the uL24 mutant (but not the uL23) \cite{Kudva2018}.

The ribosomal tunnel also speeds up elongation of neutral and negatively-charged peptides.
This is attributed to the sporadic negative patches within the ribosomal exit tunnel \cite{Lu2008}.

The ribosome clearly has the potential to prefold motifs and small domains before translocon insertion.

\section{Methods}
\subsection{Datasets}
\subsection{Complexity}
\subsection{Statistics}

\section{Results}
\subsection{There are step changes in TMH complexity depending on the TMH number in GPCRs}
GPCR distribution tables for complexity and hydrophobicity

Graphs of complexity and hydrophobicity distributions

Show there are step changes in GPCRs from Bahadur

Supplementary tables for additional stats tests and hydrophobicities

\subsection{Complexity ascention repeats according to how many TM-bundles are in the protein.}
GPCR distribution tables for complexity and hydrophobicity
Graphs of complexity and hydrophobicity distributions
Show there are step changes in GPCRs from Bahadur
Supplementary tables for additional stats tests and hydrophobicities

\subsection{The pattern is present for GPCR subfamilies}
Figure of complexity distributions with Rhodopsin like, Secretin, metabotropic glutamate, Fungal mating, cyclic AMP, Frizzled and smooth.
Bahadur tables also

\subsection{The prevelance of this amongst all TMPs.}
Mechano-sensitive (controlled vocabulary if no list available) distributions
Voltage gated (controlled vocabulary if no list available) distributions

\section{Discussion}
GPCRs have long be known to be overrepresented among genomes \cite{Remm2000}.

Seen across a variety of 7TM families with varying functions with datasets built from all membrane types (hence variety).
Suggests a pressure for simpler TMHs to precede more complex ones, repeating every 3-4 TMHs.
The universality points toward translocon behaviour pressure, or thermodynamic stability in the membrane.
Would we expect this behaviour if the translocon acted on only one TMH at a time?
