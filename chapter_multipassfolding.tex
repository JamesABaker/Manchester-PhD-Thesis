
\chapter{Co-operative TMH Insertion}
\section{Abstract}
\section{Introduction}

Steves paper GPCRs (opsin) helix 4 5 6 are folded together - Ismail, N., Crawshaw, S. G., Cross, B. C. S., Haagsma, A. C., \& High, S. (2008). Specific transmembrane segments are selectively delayed at the ER translocon during opsin biogenesis. The Biochemical Journal, 411(3), 495–506. https://doi.org/10.1042/BJ20071597



Good bad graphic from von heijne group association can happen in tunnel, in translocon, or after insertion Cymer, F., von  Heijne, G., \& White, S. H. (2015). Mechanisms of Integral Membrane Protein Insertion and Folding. Journal of Molecular Biology, 427(5), 999–1022. https://doi.org/10.1016/J.JMB.2014.09.014



Potassium channel (shaker family) S3–S4 insertion may, in part, occur in a sequential or “cotranslational” manner potassium channel - Zhang,L., Sato,Y., Hessa,T., von Heijne,G., Lee,J.-K., Kodama,I., Sakaguchi,M. and Uozumi,N. (2007) Contribution of hydrophobic and electrostatic interactions to the membrane integration of the Shaker K+ channel voltage sensor domain. Proc. Natl. Acad. Sci. U. S. A., 104, 8263–8.




SecY “cracks” open according to crystal structure analysis. Allows potential TMH to probe the environment. Egea,P.F. and Stroud,R.M. Lateral opening of a translocon upon entry of protein suggests the mechanism of insertion into membranes. 10.1073/pnas.1012556107.




Translational Arrest Peptides as in Vivo Force Sensors found interactions between C-terminal TMH and upstream TMH as the C-terminal TMH is partitioning from the translocon. 1. Cymer,F. and von Heijne,G. (2013) Cotranslational folding of membrane proteins probed by arrest-peptide-mediated force measurements. Proc. Natl. Acad. Sci. U. S. A., 110, 14640–5.

\subsection{Ribsomes in the biogenesis of membrane proteins.}
Accessibility assay and an improved intramolecular crosslinking assay showed that the helical transmembrane S3b–S4 hairpin (“paddle”) of a voltage-gated potassium (Kv)forms in the ribosome tunnel - Tu,L., Khanna,P. and Deutsch,C. (2014) Transmembrane Segments Form Tertiary Hairpins in the Folding Vestibule of the Ribosome. J. Mol. Biol., 426, 185–198.

 Size dependent folding happens through the exit tunnel. Ribosome mutants (Leucine 23 and 24 deletions) Zinc finger folds deeper in L23 mutant than wildtype (but not L24) and a 100 residue domain folds deeper than the L24 mutant (but not the L23). Kudva,R., Pardo-Avila,F., Sandhu,H., Carroni,M., Bernstein,H.D. and Heijne,G. Von (2018) The Shape of the Ribosome Exit Tunnel Affects Cotranslational Protein Folding. bioRxiv, 10.1101/274191.

Ribosomal folding of the TMHs (in Kv1.3) is kept in the translocon therefor tertiary folding of the voltage sensor domain occurs via preformed secondary-structure formation. Tu,L.W. and Deutsch,C. (2010) A Folding Zone in the Ribosomal Exit Tunnel for Kv1.3 Helix Formation. J. Mol. Biol., 396, 1346–1360.

Ribosomal tunnel speeds up elongation of neutral and negatively-charged peptides because it has sporadic negative patches. Lu,J. and Deutsch,C. (2008) Electrostatics in the Ribosomal Tunnel Modulate Chain Elongation Rates. J. Mol. Biol., 384, 73–86.



\section{Methods}
\subsection{Datasets}
\subsection{Complexity}
\subsection{Statistics}

\section{Results}
\subsection{There are step changes in TMH complexity depending on the TMH number in GPCRs}
GPCR distribution tables for complexity and hydrophobicity

Graphs of complexity and hydrophobicity distributions

Show there are step changes in GPCRs from Bahadur

Supplementary tables for additional stats tests and hydrophobicities

\subsection{Complexity ascention repeats according to how many TM-bundles are in the protein.}
GPCR distribution tables for complexity and hydrophobicity
Graphs of complexity and hydrophobicity distributions
Show there are step changes in GPCRs from Bahadur
Supplementary tables for additional stats tests and hydrophobicities


\subsection{The pattern is present for GPCR subfamilies}
Figure of complexity distributions with Rhodopsin like, Secretin, metabotropic glutamate, Fungal mating, cyclic AMP, Frizzled and smooth.
Bahadur tables also


\subsection{The prevelance of this amongst all TMPs.}
Mechano-sensitive (controlled vocabulary if no list available) distributions
Voltage gated (controlled vocabulary if no list available) distributions


\section{Discussion}
GPCRs have long be known to be overrepresented among genomes \cite{Remm2000}.

Seen across a variety of 7TM families with varying functions with datasets built from all membrane types (hence variety).
Suggests a pressure for simpler TMHs to precede more complex ones, repeating every 3-4 TMHs.
The universality points toward translocon behaviour pressure, or thermodynamic stability in the membrane.
Would we expect this behaviour if the translocon acted on only one TMH at a time?
