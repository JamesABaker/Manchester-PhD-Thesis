
\chapter{Identifying Intramembrane Folds Using Sequence Complexity} %Perhaps
\section{Abstract}
\section{Introduction}
\section{Methods}
\subsection{Datasets}
\subsection{Complexity}
\subsection{Statistics}

\section{Results}
\subsection{There are step changes in TMH complexity depending on the TMH number in GPCRs}
GPCR distribution tables for complexity and hydrophobicity

Graphs of complexity and hydrophobicity distributions

Show there are step changes in GPCRs from Bahadur

Supplementary tables for additional stats tests and hydrophobicities

\subsection{Complexity ascention repeats according to how many TM-bundles are in the protein.}
GPCR distribution tables for complexity and hydrophobicity
Graphs of complexity and hydrophobicity distributions
Show there are step changes in GPCRs from Bahadur
Supplementary tables for additional stats tests and hydrophobicities


\subsection{The pattern is present for GPCR subfamilies}
Figure of complexity distributions with Rhodopsin like, Secretin, metabotropic glutamate, Fungal mating, cyclic AMP, Frizzled and smooth.
Bahadur tables also


\subsection{The prevelance of this amongst all TMPs.}
Mechano-sensitive (controlled vocabulary if no list available) distributions
Voltage gated (controlled vocabulary if no list available) distributions


\section{Discussion}
GPCRs have long be known to be overrepresented among genomes \cite{Remm2000}.

Seen across a variety of 7TM families with varying functions with datasets built from all membrane types (hence variety).
Suggests a pressure for simpler TMHs to precede more complex ones, repeating every 3-4 TMHs.
The universality points toward translocon behaviour pressure, or thermodynamic stability in the membrane.
Would we expect this behaviour if the translocon acted on only one TMH at a time?
